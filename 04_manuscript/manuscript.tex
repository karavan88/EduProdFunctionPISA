% Options for packages loaded elsewhere
\PassOptionsToPackage{unicode}{hyperref}
\PassOptionsToPackage{hyphens}{url}
\PassOptionsToPackage{dvipsnames,svgnames,x11names}{xcolor}
%
\documentclass[
]{article}

\usepackage{amsmath,amssymb}
\usepackage{iftex}
\ifPDFTeX
  \usepackage[T1]{fontenc}
  \usepackage[utf8]{inputenc}
  \usepackage{textcomp} % provide euro and other symbols
\else % if luatex or xetex
  \usepackage{unicode-math}
  \defaultfontfeatures{Scale=MatchLowercase}
  \defaultfontfeatures[\rmfamily]{Ligatures=TeX,Scale=1}
\fi
\usepackage{lmodern}
\ifPDFTeX\else  
    % xetex/luatex font selection
\fi
% Use upquote if available, for straight quotes in verbatim environments
\IfFileExists{upquote.sty}{\usepackage{upquote}}{}
\IfFileExists{microtype.sty}{% use microtype if available
  \usepackage[]{microtype}
  \UseMicrotypeSet[protrusion]{basicmath} % disable protrusion for tt fonts
}{}
\makeatletter
\@ifundefined{KOMAClassName}{% if non-KOMA class
  \IfFileExists{parskip.sty}{%
    \usepackage{parskip}
  }{% else
    \setlength{\parindent}{0pt}
    \setlength{\parskip}{6pt plus 2pt minus 1pt}}
}{% if KOMA class
  \KOMAoptions{parskip=half}}
\makeatother
\usepackage{xcolor}
\usepackage[top=30mm,left=30mm,right=30mm]{geometry}
\setlength{\emergencystretch}{3em} % prevent overfull lines
\setcounter{secnumdepth}{5}
% Make \paragraph and \subparagraph free-standing
\makeatletter
\ifx\paragraph\undefined\else
  \let\oldparagraph\paragraph
  \renewcommand{\paragraph}{
    \@ifstar
      \xxxParagraphStar
      \xxxParagraphNoStar
  }
  \newcommand{\xxxParagraphStar}[1]{\oldparagraph*{#1}\mbox{}}
  \newcommand{\xxxParagraphNoStar}[1]{\oldparagraph{#1}\mbox{}}
\fi
\ifx\subparagraph\undefined\else
  \let\oldsubparagraph\subparagraph
  \renewcommand{\subparagraph}{
    \@ifstar
      \xxxSubParagraphStar
      \xxxSubParagraphNoStar
  }
  \newcommand{\xxxSubParagraphStar}[1]{\oldsubparagraph*{#1}\mbox{}}
  \newcommand{\xxxSubParagraphNoStar}[1]{\oldsubparagraph{#1}\mbox{}}
\fi
\makeatother


\providecommand{\tightlist}{%
  \setlength{\itemsep}{0pt}\setlength{\parskip}{0pt}}\usepackage{longtable,booktabs,array}
\usepackage{calc} % for calculating minipage widths
% Correct order of tables after \paragraph or \subparagraph
\usepackage{etoolbox}
\makeatletter
\patchcmd\longtable{\par}{\if@noskipsec\mbox{}\fi\par}{}{}
\makeatother
% Allow footnotes in longtable head/foot
\IfFileExists{footnotehyper.sty}{\usepackage{footnotehyper}}{\usepackage{footnote}}
\makesavenoteenv{longtable}
\usepackage{graphicx}
\makeatletter
\def\maxwidth{\ifdim\Gin@nat@width>\linewidth\linewidth\else\Gin@nat@width\fi}
\def\maxheight{\ifdim\Gin@nat@height>\textheight\textheight\else\Gin@nat@height\fi}
\makeatother
% Scale images if necessary, so that they will not overflow the page
% margins by default, and it is still possible to overwrite the defaults
% using explicit options in \includegraphics[width, height, ...]{}
\setkeys{Gin}{width=\maxwidth,height=\maxheight,keepaspectratio}
% Set default figure placement to htbp
\makeatletter
\def\fps@figure{htbp}
\makeatother
% definitions for citeproc citations
\NewDocumentCommand\citeproctext{}{}
\NewDocumentCommand\citeproc{mm}{%
  \begingroup\def\citeproctext{#2}\cite{#1}\endgroup}
\makeatletter
 % allow citations to break across lines
 \let\@cite@ofmt\@firstofone
 % avoid brackets around text for \cite:
 \def\@biblabel#1{}
 \def\@cite#1#2{{#1\if@tempswa , #2\fi}}
\makeatother
\newlength{\cslhangindent}
\setlength{\cslhangindent}{1.5em}
\newlength{\csllabelwidth}
\setlength{\csllabelwidth}{3em}
\newenvironment{CSLReferences}[2] % #1 hanging-indent, #2 entry-spacing
 {\begin{list}{}{%
  \setlength{\itemindent}{0pt}
  \setlength{\leftmargin}{0pt}
  \setlength{\parsep}{0pt}
  % turn on hanging indent if param 1 is 1
  \ifodd #1
   \setlength{\leftmargin}{\cslhangindent}
   \setlength{\itemindent}{-1\cslhangindent}
  \fi
  % set entry spacing
  \setlength{\itemsep}{#2\baselineskip}}}
 {\end{list}}
\usepackage{calc}
\newcommand{\CSLBlock}[1]{\hfill\break\parbox[t]{\linewidth}{\strut\ignorespaces#1\strut}}
\newcommand{\CSLLeftMargin}[1]{\parbox[t]{\csllabelwidth}{\strut#1\strut}}
\newcommand{\CSLRightInline}[1]{\parbox[t]{\linewidth - \csllabelwidth}{\strut#1\strut}}
\newcommand{\CSLIndent}[1]{\hspace{\cslhangindent}#1}

\usepackage{booktabs}
\usepackage{caption}
\usepackage{longtable}
\usepackage{colortbl}
\usepackage{array}
\usepackage{anyfontsize}
\usepackage{multirow}
\makeatletter
\@ifpackageloaded{caption}{}{\usepackage{caption}}
\AtBeginDocument{%
\ifdefined\contentsname
  \renewcommand*\contentsname{Table of contents}
\else
  \newcommand\contentsname{Table of contents}
\fi
\ifdefined\listfigurename
  \renewcommand*\listfigurename{List of Figures}
\else
  \newcommand\listfigurename{List of Figures}
\fi
\ifdefined\listtablename
  \renewcommand*\listtablename{List of Tables}
\else
  \newcommand\listtablename{List of Tables}
\fi
\ifdefined\figurename
  \renewcommand*\figurename{Figure}
\else
  \newcommand\figurename{Figure}
\fi
\ifdefined\tablename
  \renewcommand*\tablename{Table}
\else
  \newcommand\tablename{Table}
\fi
}
\@ifpackageloaded{float}{}{\usepackage{float}}
\floatstyle{ruled}
\@ifundefined{c@chapter}{\newfloat{codelisting}{h}{lop}}{\newfloat{codelisting}{h}{lop}[chapter]}
\floatname{codelisting}{Listing}
\newcommand*\listoflistings{\listof{codelisting}{List of Listings}}
\makeatother
\makeatletter
\makeatother
\makeatletter
\@ifpackageloaded{caption}{}{\usepackage{caption}}
\@ifpackageloaded{subcaption}{}{\usepackage{subcaption}}
\makeatother

\ifLuaTeX
  \usepackage{selnolig}  % disable illegal ligatures
\fi
\usepackage{bookmark}

\IfFileExists{xurl.sty}{\usepackage{xurl}}{} % add URL line breaks if available
\urlstyle{same} % disable monospaced font for URLs
\hypersetup{
  pdftitle={Estimating Cognitive Skill Formation in Brazil, China, and Russia: An Education Production Function Approach},
  pdfauthor={Garen Avanesian},
  colorlinks=true,
  linkcolor={blue},
  filecolor={Maroon},
  citecolor={Blue},
  urlcolor={Blue},
  pdfcreator={LaTeX via pandoc}}


\title{Estimating Cognitive Skill Formation in Brazil, China, and
Russia: An Education Production Function Approach}
\author{Garen Avanesian}
\date{September 17, 2024}

\begin{document}
\maketitle

\renewcommand*\contentsname{Table of contents}
{
\hypersetup{linkcolor=}
\setcounter{tocdepth}{3}
\tableofcontents
}

\textbf{Abstract}

This study examines cognitive skill formation in secondary education
across Brazil, Russia, and selected Chinese administrative units using
PISA 2018 data. It applies a value-added estimation strategy, using
grade progression as a proxy for annual learning gains and controlling
for age--grade effects through random intercepts. A mixed-effects
framework decomposes variation at the individual and school levels by
country/territory. Results show substantial cross-country heterogeneity
in the learning gains, which show little relation to the overall volume
of learning per territory. Between-school variance ranges from 15\% to
over 50\% depending on the territory, reflecting systemic differences in
the role of schools in shaping cognitive skills. Among individual
characteristics, SES exerts the largest effect, while gender and
language minority effects are smaller yet substantial. Peer effects
approximated by the socio-economic composition of students at school are
as strong as one year of schooling at lowest, underscoring the role of
external environment in shaping cognitive skills. Task mastery also
strongly predicts achievement, consistent with the ``skills beget
skills'' framework, when non-cognitive skills accelerate cognitive ones.
The study concludes by acknowledging that institutional structure of
education systems and peer composition of schools jointly drive
educational inequality.

\textbf{Keywords:} cognitive skills, education production function,
human capital, non-cognitive skills, socio-economic status,
mixed-effects models

\textbf{JEL Codes:} I24, J24, C51

\newpage

\section{Introduction}\label{introduction}

The acquisition of skills and knowledge through formal education is a
multifaceted phenomenon that has garnered attention from various
academic disciplines, each offering unique perspectives on the issue.
For instance, psychology delves into the individual aspects of learning,
such as cognitive processes and motivation. Sociology, on the other
hand, investigates the institutional determinants and the influence of
social environments on educational outcomes. Economics, in turn,
typically evaluates the efficiency of the education system, positing
that a system is more effective when it equips students with skills and
knowledge needed to compete in the labor market and succeed in life in a
sustainable way.

Historically, the quality of education was not a primary focus of
economic analysis. Instead, the human capital of individuals was
evaluated based on the years of formal education or the possession of
certified qualifications. This approach is exemplified by the Mincerian
wage equation, which models the returns to education by considering
educational attainment as the main variable (Mincer, 1958). However,
this model implicitly assumes that individuals with equivalent levels of
schooling or similar diplomas possess identical skills and knowledge,
thereby overlooking individual differences in personality and abilities.
As noted, attainment as an outcome of the education process ``assumes a
year of schooling produces the same amount of student achievement, or
skills, over time and in every country'' and ``simply counts the time
spent in schools without judging what happens in schools'' (Hanushek,
2020). Needless to say, people who graduate from the same schools and
spend the same number of years in formal education are quite
heterogeneous in terms of abilities and, as a consequence, labor
productivity.

Recognizing this limitation led to a significant shift in economic
research. Instead of just measuring how much education in terms of years
people have, economists focused on the quality of learning and skills,
starting from the very early years in formal education. Accordingly, the
recent global development agenda in education, encapsulated by
Sustainable Development Goal 4 (SDG4), includes among its targets the
measurement of the proportion of children achieving minimum proficiency
levels in reading and mathematics at key stages: early grades (2 and 3),
the end of primary, and lower secondary education (targets 4.1.1a,
4.1.1b, and 4.1.1c respectively). This perspective underscores that
skills and knowledge are not merely byproducts of the educational
process but are crucial indicators of its quality. The effectiveness of
an education system, therefore, hinges on its capacity to foster these
outcomes and in a broader sense facilitate sustainable economic
development (Altinok \& Aydemir, 2017; Hanushek \& Woessmann, 2007,
2008, 2015; Vittadini et al., 2022).

Educational outcomes are shaped by a myriad of inputs, both endogenous
(such as school infrastructure, curricula, and teacher qualifications)
and exogenous (including parental education and occupation, household
characteristics, and individual personality traits). While policy impact
on exogenous factors is either very limited or impossible, endogenous
factors that constitute the institutional environment of skill
acquisition can be directly or indirectly influenced by policymakers,
highlighting the role of strategic educational policy in enhancing the
quality of learning (Hanushek, 2020). To estimate how inputs of the
education process relate to the outcome, namely, cognitive skills,
economists conventionally use education production functions (Bowles \&
Levin, 1968; Brown \& Saks, 1975; Hanushek, 1979, 1987, 2020; Monk,
1989). As noted, ``{[}s{]}tudies of educational production functions
examine the relationship among the different inputs into the educational
process and outcomes of the process'' by typically employing
``statistical techniques, quite commonly some form of regression
analysis, to separate the effects of different inputs and to estimate
the magnitude or significance of any relationships'' (Hanushek, 1987).
In other words, regression models are used to predict the effect of
exogenous and endogenous factors on cognitive skills.

In the education production framework, learning is the output of the
education process, i.e., what is being produced. It is approximated by
the achievement scores derived from learning assessments. The emergence
of such large-scale learning assessments as Programme for International
Student Assessment (PISA) carried out by Organisation for Economic
Co-operation and Development (OECD), or Trends in International
Mathematics and Science Study (TIMSS) and Progress in International
Reading Literacy Study (PIRLS) carried out by International Association
for the Evaluation of Educational Achievement (IEA) produced a wealth of
internationally comparable data on learning and factors associated with
it, also serving as a source to monitor the targets of SDG4.1.1.
Normally, test scores from these assessments are used as the outputs in
estimating education production function. However, early works on
education production functions were substantially challenged by
extensive discussions with respect to the inputs of the model, as
economists emphasized the lack of theory of learning to guide the
mathematical specification of the function (Bowles, 1970; Hanushek,
1979; McClung, 1977; Simmons \& Alexander, 1978). In order to estimate
education production function and isolate the effects of other factors,
Bowles (1970) proposed the following specification of the environmental
influences on learning: (1) home, (2) community, (3) peer groups, and
(4) school. As a part of three factors that constitute non-school
environment, Bowles emphasized the role of achievement motivation as a
largely exogenous factor of learning.

Home environment refers to parental education, parental occupation,
household income, cultural goods or books, material possessions, and all
the other factors that shape a standard of living and define a
socio-economic status (SES) of a child's family. They previously were
defined as exogenous factors of education production. In a broad sense,
peer and community effects ``can best be described as externalities in
the production of cognitive skills or human capital'' (Dannemann, 2019,
p. 6). They ``encompass nearly any externality in which peers'
backgrounds, current behavior, or outcomes affect an outcome''
(Sacerdote, 2011, p. 250). These peer effects normally refer to the
aggregated values of achievement at the class or school level, aiming to
estimate how the peers' outcome affects the individual outcome. However,
peer backgrounds' characteristics, such as an aggregated value of SES,
also can be used. Accounting for the peer effects in the education
production functions is necessary in the light of policy decisions: the
existing evidence suggests that the magnitude and character of peer
effects in education could inform the distribution of students by
schools and have critical implications on tuition fees and resource
allocation (Epple et al., 2000; Epple et al., 2004; Epple \& Romano,
2000; Epple \& Romano, 1998).

However, particular consideration in that regard should be given to the
school environment as schools, being the place where a child learns, in
this context serve as ``factories'' of educational production. In this
context, Bowles (1970) outlined 4 core dimensions of the school
environment: (1) qualification of teachers (teacher quality), (2) number
of available teaching staff (teacher quantity), (3) school policy, and
(4) infrastructure (physical facilities).

In the realm of educational economics, two primary estimation approaches
of education production are distinguished: value-added and level
estimation (Hanushek, 1987). Level estimation is applied when academic
achievement is measured at a single time point, with the econometric
model designed to discern the overall average impacts of educational
inputs on outcomes. In other words, in the terminology of Bowles (1970),
in this case ``an achievement score must be considered a measure of
gross output'' (p.~26) of the learning production. Conversely, the
value-added approach is utilized when assessing the relationship between
educational inputs and student achievement across two time points. This
method emphasizes the educational progress occurring between these
intervals, thereby allowing for an analysis of the incremental learning
attributable to specific educational inputs. Essentially, value-added
models quantify the learning gains acquired within a defined timeframe,
such as the academic knowledge an average student gains over a school
year, while adjusting for their baseline achievement (i.e., innate
ability). In other words, it treats the achievement score as a net
output of the education process (Bowles, 1970). In contrast, level
estimation predicts the average influence of educational inputs on
cognitive achievement without accounting for individual learning
progression over time, treating innate ability as part of the error term
in the model. Therefore, education production functions that employ
value-added models are particularly adept at evaluating the efficiency
of national education systems. They facilitate the identification of
inputs that significantly enhance learning gains over time. Notably,
when such analyses reveal significant effects on productivity linked to
endogenous factors, particularly those at the school level, they can
lead to substantive policy recommendations (Khan \& Kiefer, 2007).

Despite being used as a means to guide education policy and increase of
school productivity, education production function has a number of
limitations. As such, the presence of endogeneity resulting from omitted
variables could substantially affect estimation of learning gains. As
was noted, in this case endogeneity arises ``from correlations between
included student characteristics and omitted school variables'', which
``are mainly the result of stratification'' (Hanchane \& Mostafa, 2012)
and inequality present in the society. In other words, economically
disadvantaged families are more likely to reside in the relatively poor
neighborhoods, and schools in these communities predominantly consist of
schoolchildren with the similar - disadvantaged - background. In turn,
some characteristics of the schools (e.g., funding, human resources,
infrastructure) may be related to the location of the school. In that
regard, school characteristics could be strongly associated with the
social status of its students. These considerations present substantial
constraints for interpreting the results of education production
functions in the causal manner. Furthermore, complex relationships
between different school inputs could be a subject of multicollinearity.
A more detailed description of the econometric shortcomings of education
production function was provided in Goldhaber \& Brewer (1997). However,
``'{[}w{]}hile the critics raise powerful arguments, policy makers
nonetheless need advice on effective ways of allocating resources'' and
therefore ``abandonment of the production function method may be too
extreme a response to its limitation'' (Khan \& Kiefer, 2007).

This study contributes to the pursuit of Sustainable Development Goal
4.1.1c, which aims to ensure that all students acquire the knowledge and
skills needed to promote sustainable development. It employs the
education production function framework to explore learning gains and
the comparative effectiveness of the education systems in Brazil,
Chinese administrative units, and Russia, countries marked by
significant economic inequality. Utilizing data from the 2018 PISA, this
analysis aims to dissect educational outcomes within these economies. As
no nationally representative sample for China is available, this
analysis covers only the cities of Beijing, Shanghai, Jiangsu, and
Zhejiang (B-S-J-Z) in mainland China, as well as special administrative
regions of Hong Kong and Macao.

Despite similar levels of economic development (in 2018, all three
countries were classified as upper-middle income ones based on their GNI
per capita), these countries exhibit pronounced disparities in
educational performance. The PISA report highlights the heterogeneity in
their educational productivity: Chinese cities remarkably outperform the
average of the OECD member states, Russia hovers just below this average
(though Moscow City is reported to stand quite above the national and
the OECD averages), and Brazil lags significantly behind (OECD, 2019b).
These variations offer fertile ground for investigating the principal
factors that contribute to learning gains among secondary school
students within a similar economic tier.

A primary research question guiding this inquiry is: What are the
principal factors contributing to learning gains among secondary school
students in Brazil, Chinese administrative units, and Russia?
Understanding the extent to which school-level factors contribute to the
variation in student performance is critical, bearing in mind the
practical implications for educational policy. This understanding allows
for assessing the sensitivity of learning outputs to institutional
interventions at the school level. Furthermore, previous research has
highlighted heterogeneity in school productivity with respect to
students' socio-economic characteristics (Gyimah-Brempong \& Gyapong,
1991; Hanushek, 1979). Acknowledging the high levels of economic
inequality in Brazil, China, and Russia prompts a critical secondary
research question incorporating an equity perspective into the education
production: Are learning gains heterogeneous with respect to students'
socio-economic status, as well as socio-economic composition of students
in school? This question arises from the premise that even the recent
results of internationally comparative large scale learning assessments
underscore that economic inequality strongly relates to educational
outcomes, resulting in significant achievement gaps (OECD, 2019b). With
that respect, it is essential to examine if and how students'
socio-economic backgrounds affect their learning progress in different
educational systems. This socio-economic background operates through two
dimensions: SES of a family to which a child belongs, and socio-economic
composition of students in schools where a child learns. This problem
statement lays a solid foundation for discerning the influence of
national educational policies and system characteristics on scholastic
achievement. The present comparative analysis aims to inform
policymakers and education stakeholders about effective approaches to
enhancing educational quality and equity in similar economic contexts.

\section{Data}\label{data}

This study utilizes the 2018 PISA data for Brazil, selected regions of
China, and Russia, providing a comprehensive assessment of student
achievement in reading, mathematics, and science, as well as information
on student and school-level factors that affect learning. As has been
mentioned, the Chinese data are not nationally representative, covering
only the cities of Beijing, Shanghai, Jiangsu, and Zhejiang (B-S-J-Z) in
mainland China. Additionally, separate assessments were conducted in the
Chinese special administrative regions of Hong Kong and Macao. In
contrast, Russia's data includes a nationally representative sample, as
well as separate samples from the city of Moscow, Moscow region, and the
Republic of Tatarstan. Brazil's sample is uniform, with no
differentiation by territorial units. A summary of the sample is
presented in Table~\ref{tbl-sample-summary}.

\textbf{\emph{Output variable}}

While PISA collects the data on functional literacy in three cognitive
domains, namely, reading, mathematics, and science, the test scores in
reading serve as the dependent variable in the study. There are several
reasons behind that. First, 2018 analytical framework of PISA defines
reading comprehension within 3 key processes, which refer to locating
information, understanding, and evaluating and reflecting. The last one
refers to the most high-level process as ``readers must go beyond
understanding the literal or inferred meaning of a piece of text or a
set of texts to assess the quality and validity of its content and
form'' (OECD, 2019b, p. 36). It consists of such cognitive processes as
(1) assessing quality and credibility, where a readers judges whether
the content is accurate, unbiased, and valid; (2) reflecting on content
and form, where a reader assesses the text on its quality and style; (3)
corroborating and handling conflict, where a reader compares information
in texts, identifies contradictions between them, and makes decisions on
how to handle these contradictions. All these cognitive processes are
also essential for skills in mathematics and science, as they are laying
the foundation for more complex cognitive processes in these two
domains. Another study, using PISA 2018, has found that ``reading and
mathematics were important predictors of science achievement, and the
effects of reading significantly exceeded mathematics'' (Zhu, 2021).

It was also highlighted based on the statistical analysis of PIRLS and
TIMMS data that achievement in all three cognitive skills is highly
correlated (Caponera et al., 2016). Section~\ref{sec-app1} of Appendix
tests this assumption and presents correlations between reading, math,
and science by country. It confirms that carrying out the analysis on
three cognitive skills separately is redundant as they are highly
correlated. As the original PISA dataset standardizes the cognitive
score at the OECD average and standard deviation, for the purpose of the
analysis the dependent variable was rescaled to reflect the average and
standard deviation of the selected sample of Brazil, Chinese cities, and
Russia. The respective probability density plots can be found in
Section~\ref{sec-app3} of Appendix.

\textbf{\emph{Input variables}}

The grade of study serves as a major predictor in the education
production function, proxying the learning gain that occurs in one
academic year and thus allowing for the estimation in a value-added
manner. Student-level characteristics, such as age and sex, are included
as socio-demographic controls. Additionally, the language spoken at home
is used to approximate minority status, given the context of three
states with different ethnic groups.

\begin{table}

\caption{\label{tbl-sample-summary}Sample Summary}

\centering{

\fontsize{12.0pt}{14.0pt}\selectfont
\begin{tabular*}{\linewidth}{@{\extracolsep{\fill}}lcccccccc}
\toprule
Variable & \textbf{Brazil}  N = 10,691\textsuperscript{\textit{1}} & \textbf{BSJZ}  N = 12,058\textsuperscript{\textit{1}} & \textbf{Hong Kong}  N = 6,037\textsuperscript{\textit{1}} & \textbf{Macao (China)}  N = 3,775\textsuperscript{\textit{1}} & \textbf{Moscow}  N = 5,768\textsuperscript{\textit{1}} & \textbf{Moscow Region}  N = 2,016\textsuperscript{\textit{1}} & \textbf{Russia}  N = 7,608\textsuperscript{\textit{1}} & \textbf{Tatarstan}  N = 5,816\textsuperscript{\textit{1}} \\ 
\midrule\addlinespace[2.5pt]
{\bfseries Age} & 15.90 (0.28) & 15.77 (0.29) & 15.73 (0.29) & 15.83 (0.29) & 15.78 (0.28) & 15.80 (0.28) & 15.81 (0.29) & 15.79 (0.28) \\ 
{\bfseries Gender} &  &  &  &  &  &  &  &  \\ 
    Female & 5,478 (51\%) & 5,775 (48\%) & 2,955 (49\%) & 1,862 (49\%) & 2,847 (49\%) & 970 (48\%) & 3,861 (51\%) & 2,906 (50\%) \\ 
    Male & 5,213 (49\%) & 6,283 (52\%) & 3,082 (51\%) & 1,913 (51\%) & 2,921 (51\%) & 1,046 (52\%) & 3,747 (49\%) & 2,910 (50\%) \\ 
{\bfseries ESCS} & -0.76 (1.10) & -0.07 (0.98) & -0.22 (0.92) & -0.21 (0.82) & 0.79 (0.58) & 0.57 (0.61) & 0.41 (0.66) & 0.41 (0.64) \\ 
{\bfseries Area} &  &  &  &  &  &  &  &  \\ 
    1. Village/Small Town & 1,681 (17\%) & 2,735 (23\%) & 134 (3.0\%) & 5 (0.1\%) & 0 (0\%) & 388 (19\%) & 1,691 (23\%) & 1,289 (22\%) \\ 
    2. Town & 3,506 (35\%) & 1,890 (16\%) & 775 (17\%) & 0 (0\%) & 39 (0.7\%) & 553 (27\%) & 1,753 (24\%) & 1,093 (19\%) \\ 
    3. City/Large City & 4,870 (48\%) & 7,433 (62\%) & 3,571 (80\%) & 3,770 (100\%) & 5,729 (99\%) & 1,075 (53\%) & 3,783 (52\%) & 3,434 (59\%) \\ 
{\bfseries Grade} &  &  &  &  &  &  &  &  \\ 
    Grade 10 & 3,430 (32\%) & 7,601 (63\%) & 4,108 (68\%) & 2,187 (58\%) & 559 (9.7\%) & 188 (9.3\%) & 809 (11\%) & 221 (3.8\%) \\ 
    Grade 11 & 4,608 (43\%) & 132 (1.1\%) & 51 (0.8\%) & 39 (1.0\%) & 6 (0.1\%) & 2 (<0.1\%) & 7 (<0.1\%) & 0 (0\%) \\ 
    Grade 12 & 219 (2.0\%) & 7 (<0.1\%) & 0 (0\%) & 1 (<0.1\%) & 0 (0\%) & 0 (0\%) & 0 (0\%) & 0 (0\%) \\ 
    Grade 7 & 378 (3.5\%) & 26 (0.2\%) & 56 (0.9\%) & 71 (1.9\%) & 14 (0.2\%) & 2 (<0.1\%) & 23 (0.3\%) & 4 (<0.1\%) \\ 
    Grade 8 & 744 (7.0\%) & 190 (1.6\%) & 315 (5.2\%) & 356 (9.4\%) & 419 (7.3\%) & 124 (6.2\%) & 591 (7.8\%) & 559 (9.6\%) \\ 
    Grade 9 & 1,312 (12\%) & 4,102 (34\%) & 1,507 (25\%) & 1,121 (30\%) & 4,770 (83\%) & 1,700 (84\%) & 6,178 (81\%) & 5,032 (87\%) \\ 
{\bfseries Language at Home} &  &  &  &  &  &  &  &  \\ 
    1. Language of the test & 10,322 (99\%) & 11,934 (100\%) & 4,914 (84\%) & 3,070 (81\%) & 5,270 (96\%) & 1,896 (95\%) & 6,867 (92\%) & 4,673 (82\%) \\ 
    2. Other language & 152 (1.5\%) & 58 (0.5\%) & 959 (16\%) & 704 (19\%) & 220 (4.0\%) & 96 (4.8\%) & 585 (7.9\%) & 1,038 (18\%) \\ 
\bottomrule
\end{tabular*}
\begin{minipage}{\linewidth}
\textsuperscript{\textit{1}}Mean (SD); n (\%)\\
\end{minipage}

}

\end{table}%

Source: Calculations by the authors based on the PISA 2018 data.

Non-cognitive characteristics, such as motivation to achieve and learn,
are also important factors influencing learning gains. Task mastery,
operationalized through the concept of achievement motivation in PISA
2018 (Buchholz et al., 2022; OECD, 2019c, 2019a), is a key predictor of
academic achievement, particularly among students from economically
disadvantaged backgrounds (Avanesian et al., 2022).

The Index of economic, social, and cultural status (ESCS) aggregates
factors like parental education, occupation, material possessions, and
cultural goods, which constitute the milieu associated with child
socialization. This composite measure is used to avoid multicollinearity
and to recognize that these parameters are exogenous and cannot be
subject to policy interventions. To achieve an equity perspective, the
index is divided into three percentile groups: bottom 40\%, middle 50\%,
and top 10\%.

The model also accounts for the peer effects by including the average
ESCS per school as a proxy of social environment. School-level factors,
such as location, type, student-teacher ratio, average class size,
teacher qualification, and poor school infrastructure, are also
represented.

All the descriptive statistics of the variables used in the study are
presented in Section~\ref{sec-app2} of Appendix. In order to have a
preliminary idea of how the inputs of the regression models are
correlated between each other, correlation plots were produced and can
be found in Section~\ref{sec-app4} of Appendix.

\section{Econometric Verification
Strategy}\label{econometric-verification-strategy}

Econometric literature suggests many different ways to operationalize
the education production function, with some variations amongst scholars
(Bowles, 1970; Glewwe \& Kremer, 2006; Todd \& Wolpin, 2003). Adapted
from Glewwe \& Kremer (2006), it can be written as follows:

\begin{equation}\phantomsection\label{eq-eq1}{A_{igt} = f_a \left(Q^{(t)}_{ig}, C^{(t)}_{ig}, H^{(t)}_{ig}\right)  }\end{equation}
where:\\
- \(A_{igt}\) is the academic achievement score for student \(i\) in
school \(g\) cumulative to the time \(t\);\\
- \(Q_{ig}\) refers to the vector characteristics of a school \(g\) in
which a child \(i\) studies cumulative to the time \(t\);\\
- \(C_{ig}\) is the vector of individual characteristics of a child
\(i\);\\
- \(H_{ig}\) is the vector of household/family characteristics
cumulative to the time \(t\).

\textbf{\emph{Estimation approach}}

As previously acknowledged, the PISA data are cross-sectional and do not
contain longitudinal learning records for individual students. However,
it is possible to estimate learning gains in a value-added manner by
utilizing the variable ``grade,'' thus predicting the average learning
that a typical student acquires in one academic year. This approach is
well-established in the economic literature (Avvisati \& Givord, 2021a,
2021b, 2023), where previous studies estimated learning gains by
accounting for the complex interplay between age, grade, test timing,
and school starting age through difference-in-difference or instrumental
variable methods. Building on similar theoretical assumptions regarding
the sources of endogenous variation, this study utilizes a different
verification model, adopting a mixed-effects (multilevel) model that has
a number of advantages.

While age is an exogenous factor, it introduces endogeneity in the
education production function due to students' differing durations in
the education system. The PISA dataset samples 15-year-olds, introducing
only a 12-month age variability, but the distribution of age by grade
spans five academic years, creating substantial variability. This makes
the relationship between age and grade a source of endogenous variation.
Students of the same age but in different grades may have substantially
different baseline achievements and subsequent learning gains due to
variations in the time spent in education. These differences may result
from either grade repetition (which is controlled for in the model) or
variations in school entry age. Therefore, as some students might be
either over-aged or under-aged for their grade, the model must isolate
the effect of age on learning.

Adopting the random intercept that occurs due to a student's age allows
for individual variations in learning outcomes based on different ages.
Without ignoring between-age variation (like it happens in conventional
econometric estimation via fixed-effect models based on de-meaning), it
separates within- and between-age variation in learning due to age by
accounting for age-related differences across students. The adopted
random intercept term captures unobserved heterogeneity associated with
age, i.e., it models the variation in learning that cannot be explained
solely by the fixed effects, while allowing for different baseline
learning across age. In essence, this approach handles both observed and
unobserved age-related variability, ensuring that the random intercept
absorbs the unmeasured effects specific to different ages, allowing the
fixed effect of grade to capture the systematic learning variations.
This strategy facilitates the estimation of grade-related learning gains
in a cross-sectional dataset, even in the absence of longitudinal
records.

However, age is not the only source of endogeneity in the model. Even
within one territory, students are clustered within schools, and the
distribution of students by schools is not random. In other words,
though there are a number of school-level factors included in the model,
they might not capture all the differences in the baseline effect of a
school on learning. Moreover, including many school-level variables
could lead to multicollinearity issues. A random intercept can help
mitigate this by capturing the variance attributable to the school
itself, which might otherwise be incorrectly attributed to the observed
school-level variables.

Further, two key assumptions outlined in the research question is that
learning gains could be heterogeneous with respect to a socio-economic
status of a child's family, or socio-economic composition of students in
school. As such, we want to test the hypothesis if the learning gain
that occurs in one academic year varies due to the differences in
economic standing - both of individuals and schools. With that respect,
incorporation of random slope of grade by percentile groups of ESCS
index per individual and averaged across students per school would allow
for addressing these research questions.

The data in this study covers three countries, with China consisting of
three different samples (BSJZ, Hong Kong, and Macao), and Russia of four
(Russian sample and separate for Moscow, Moscow region, and Tatarstan).
The model needs to account for differences in schooling across these
territories, national education systems, as well as potential variations
in subnational regulations. With that respect, as each territory
provides a sufficient number of observations, mixed-effects models are
estimated separately.

Proceeding from that, the following mixed-effects model is calculated:

\begin{equation}\phantomsection\label{eq-eq2}{
CS_{ij} = \beta_0 + \beta_1 \cdot \text{Grade}_{ij} + \sum_{k \in C} \beta_{C_k} \cdot \text{C}_{ij,k} + \sum_{l \in H} \beta_{H_l} \cdot \text{H}_{ij,l} + \sum_{m \in Q} \beta_{Q_m} \cdot \text{Q}_{ij,m} \\
+ u_{0j} + (\beta_{\text{Grade}} + u_{1j}) \cdot \text{Grade}_{ij} + \epsilon_{ij},
}\end{equation} where:\\
- \(CS_{ij}\) is the cognitive skills in reading for student \(i\) in
school \(j\).\\
- \(\beta_0\) is the overall intercept. - \(\beta_1 \text{Grade}_{ij}\)
is the learning gain for a student \(i\) in school \(j\).\\
- \(\sum\limits_{k \in C} \beta_{C_k} \cdot \text{C}_{ij,k}\) represents
the sum over individual factors \((C)\), where \(\beta_{C_k}\) are
coefficients for individual variables \(\text{C}_{ij,k}\).\\
- \(\sum\limits_{l \in H} \beta_{H_l} \cdot \text{H}_{ij,l}\) represents
the sum over household factors \((H)\), where \(\beta_{H_l}\) are
coefficients for household variables \(\text{H}_{ij,l}\).\\
- \(\sum\limits_{m \in Q} \beta_{Q_m} \cdot \text{Q}_{ij,m}\) represents
the sum over school-related factors \((Q)\), where \(\beta_{Q_m}\) are
coefficients for school-related variables \(\text{Q}_{ij,m}\).\\
- \(u_{0j}\) is the random intercept for school \(j\).\\
- \(u_{1d}\) is the random slope for Grade by country \(d\).\\
- \(\epsilon_{ij}\) is the error term for student \(i\) in school \(j\).

In addition to the baseline model, two supplementary mixed-effects
models are calculated, with minor modifications of the baseline model.
The second model excludes the fixed effect of SES and average SES per
school, and instead adopt random slope of grade by country/territory and
SES percentile group. The third model also replicates the first one, but
instead of random slope of grade by individual SES percentile group, it
does the same for schools, allowing to estimate how learning gains are
different across schools varying in the concentration of students from
poor or wealthy families.

The analysis was carried out in R (R Core Team, 2021), open-source
software for statistical computing. The mixed-effects regression models
were calculated using the lme4 package (Bates et al., 2015).
Significance values for the models were produced with the help of the
lmerTest package.

\section{Results}\label{results}

\subsection{Gross Educational
Productivity}\label{gross-educational-productivity}

A descriptive exploration reveals significant variation in gross
educational productivity across countries and territories, with average
skill levels highlighting heterogeneity within the sample.
Table~\ref{tbl-desc-stats} presents data on average learning volumes by
country, further disaggregated by socio-economic status percentile
groups. The results show that the Chinese cities of Beijing, Shanghai,
Jiangsu, and Zhejiang (BSJZ) exhibit the highest average learning
volumes, with an average value of 0.60 standard deviations above the
sample mean. Following at a considerable distance are Hong Kong (0.28)
and Macao (0.26), while Moscow City records a value of 0.34. In
contrast, Brazil reports the lowest average value (--0.80), followed by
Tatarstan (--0.32) and the Russian Federation overall (--0.17). The
Moscow Region also scores slightly below the sample mean (--0.11).

Additionally, all countries and territories demonstrate substantial
disparities in learning volumes based on SES. Children from the bottom
40\% of families, in terms of SES, consistently lag behind their
higher-SES peers, with the largest gap observed in Brazil. Notably, the
difference between the bottom 40\% and the top 10\% is also pronounced
in BSJZ and Hong Kong, while it is somewhat narrower in Moscow City and
Macao. However, since these figures are influenced by multiple factors,
the following sections of the paper explore in greater depth the
underlying drivers behind the patterns identified in this descriptive
analysis.

\begin{table}

\caption{\label{tbl-desc-stats}Gross Educational Productivity (Average
Reading Score) per Country/Territory and SES}

\centering{

\fontsize{12.0pt}{14.0pt}\selectfont
\begin{tabular*}{\linewidth}{@{\extracolsep{\fill}}crrrr}
\toprule
Country/Territory & Total & Bottom 40\% & Middle 50\% & Top 10\% \\ 
\midrule\addlinespace[2.5pt]
Brazil & -0.80 & -1.08 & -0.70 & 0.02 \\ 
Hong Kong & 0.28 & 0.12 & 0.37 & 0.64 \\ 
Macao & 0.26 & 0.15 & 0.30 & 0.49 \\ 
B-S-J-Z (China) & 0.60 & 0.29 & 0.75 & 1.15 \\ 
Moscow City (RUS) & 0.34 & 0.18 & 0.48 & 0.54 \\ 
Moscow Region (RUS) & -0.11 & -0.30 & 0.01 & 0.08 \\ 
Tatarstan (RUS) & -0.32 & -0.55 & -0.16 & -0.13 \\ 
Russian Federation & -0.17 & -0.43 & 0.00 & 0.08 \\ 
\bottomrule
\end{tabular*}
\begin{minipage}{\linewidth}
Source: Calculations by the authors based on the PISA 2018 data.\\
\end{minipage}

}

\end{table}%

\subsection{Variation in Skills due to School-Level
Factors}\label{variation-in-skills-due-to-school-level-factors}

Understanding the role that schools play in shaping student learning
outcomes is not just an econometric exercise; it is a policy imperative.
This information is summarized in the Intraclass Correlation
Coefficients (ICC), which represent the variance attributable to the
random intercept terms incorporated into the model. The baseline models
for each of the territories include only reading score as a dependent
variable and a school identification as a random intercept term. In
other words, these can also be called null models. By estimating the
ICCs from the null models, we can identify how much of the variation in
students' reading scores is attributable to differences between schools.
In simpler terms, the ICC tells us how much it matters which school a
child attends, as opposed to their family SES background or individual
characteristics. The variance components of school identifiers from the
baseline mixed-effects models by each territory are presented in
Figure~\ref{fig-mem-base-icc}.

\begin{figure}

\centering{

\includegraphics{manuscript_files/figure-pdf/fig-mem-base-icc-1.pdf}

}

\caption{\label{fig-mem-base-icc}Share of variance in reading score
attributable to school-level factors, results of mixed-effects models
with random intercept terms by school identifiers, separately by each
territory}

\end{figure}%

The results show that schools do matter, but not equally across
socio-economic contexts. In places like B-S-J-Z (China) and Brazil,
school-level differences account for nearly 50\% of the variation in
reading performance. This suggests that improving the quality of schools
in these systems could have a major impact on learning outcomes. In
contrast, Hong Kong, Macao, and Russia fall into a middle category, with
school effects explaining around 30--35\% of the variance. While school
quality is still important in these cases, it may be only one part of a
broader picture that includes home environments, peer effects, and other
determinants beyond formal education.

Interestingly, administrative units within Russia, namely, Moscow City,
Moscow Region, and Tatarstan, show a much smaller share of variance
explained by schools. For instance, in the city of Moscow, only about
14\% of the variability in reading scores is attributable to
school-level factors. This presents a paradox: despite Moscow's high
average PISA performance, which lies substantially above the average of
OECD countries, the role of the school itself appears very limited. One
possibility is that much of the learning advantage in Moscow is driven
by factors outside the classroom, such as private tutoring or parental
education and occupation, or any other set of factors linked to the
socio-economic advantage. Another explanation could be that the quality
of schooling is relatively uniform across the city, which in turn limits
the observable differences between the schools. Either way, these
findings complicate the assumption that high-performing systems always
owe their success to the strength of their schools.

\subsection{Heterogeneity in Net Educational Productivity: Countries
Differ by Learning Gains}\label{sec-learn-gains}

While null models allow for estimating variance attributable to schools,
they do not outline the magnitude of specific factors behind the most
productive education systems. For this purpose, the main model with
covariates representing school, individual, and family factors, was
carried out for each territory separately. The results of the
mixed-effects models are presented in Table~\ref{tbl-reg-mem}. The
estimated learning gains associated with progressing one school grade
vary considerably across countries and territories, indicating stark
differences in the net productivity of education systems as opposed to
their gross productivity. These coefficients, interpreted as the average
increase in standard deviation of reading performance from an additional
year of schooling, range from as low as 0.203 in Hong Kong to as high as
0.375 in Moscow Region.

\begin{table}

\caption{\label{tbl-reg-mem}Summary of Mixed-Effects Model Regressions}

\centering{

\centering
\begin{talltblr}[         %% tabularray outer open
entry=none,label=none,
note{}={  },
]                     %% tabularray outer close
{                     %% tabularray inner open
colspec={Q[]Q[]Q[]Q[]Q[]Q[]Q[]Q[]Q[]},
hline{2}={1-9}{solid, black, 0.05em},
hline{34}={1-9}{solid, black, 0.05em},
hline{1}={1-9}{solid, black, 0.1em},
hline{37}={1-9}{solid, black, 0.1em},
column{2-9}={}{halign=c},
column{1}={}{halign=l},
}                     %% tabularray inner close
& Russia & Moscow City & Moscow Region & Tatarstan & BSJZ & Hong Kong & Macao & Brazil \\
(Intercept) & \num{-2.352} & \num{-3.200} & \num{-3.211} & \num{-2.637} & \num{-4.372} & \num{-4.020} & \num{-2.680} & \num{-3.892} \\
& (\num{0.512}) *** & (\num{1.109}) ** & (\num{1.563}) * & (\num{0.490}) *** & (\num{0.344}) *** & (\num{0.933}) *** & (\num{0.736}) *** & (\num{0.275}) *** \\
School year & \num{0.264} & \num{0.347} & \num{0.376} & \num{0.240} & \num{0.326} & \num{0.202} & \num{0.328} & \num{0.325} \\
& (\num{0.027}) *** & (\num{0.036}) *** & (\num{0.054}) *** & (\num{0.033}) *** & (\num{0.018}) *** & (\num{0.028}) *** & (\num{0.017}) *** & (\num{0.013}) *** \\
Sex: Male & \num{-0.167} & \num{-0.130} & \num{-0.248} & \num{-0.208} & \num{-0.066} & \num{-0.120} & \num{-0.033} & \num{-0.090} \\
& (\num{0.020}) *** & (\num{0.030}) *** & (\num{0.040}) *** & (\num{0.023}) *** & (\num{0.011}) *** & (\num{0.038}) ** & (\num{0.027}) & (\num{0.018}) *** \\
SES: Middle 50\% & \num{0.155} & \num{0.251} & \num{0.142} & \num{0.166} & \num{0.088} & \num{-0.030} & \num{0.046} & \num{0.103} \\
& (\num{0.023}) *** & (\num{0.033}) *** & (\num{0.044}) ** & (\num{0.025}) *** & (\num{0.013}) *** & (\num{0.040}) & (\num{0.026}) + & (\num{0.021}) *** \\
SES: Top 10\% & \num{0.203} & \num{0.190} & \num{0.194} & \num{0.099} & \num{0.195} & \num{-0.023} & \num{0.086} & \num{0.170} \\
& (\num{0.040}) *** & (\num{0.056}) *** & (\num{0.075}) ** & (\num{0.045}) * & (\num{0.023}) *** & (\num{0.065}) & (\num{0.045}) + & (\num{0.037}) *** \\
Language minority & \num{-0.282} & \num{-0.473} & \num{-0.424} & \num{-0.128} & \num{-0.477} & \num{-0.026} & \num{-0.613} & \num{-0.221} \\
& (\num{0.047}) *** & (\num{0.081}) *** & (\num{0.097}) *** & (\num{0.034}) *** & (\num{0.079}) *** & (\num{0.055}) & (\num{0.046}) *** & (\num{0.083}) ** \\
Area: Town & \num{0.186} &  & \num{-0.226} & \num{0.156} &  &  &  & \num{0.006} \\
& (\num{0.088}) * &  & (\num{0.191}) & (\num{0.107}) &  &  &  & (\num{0.059}) \\
Area: City/Large City & \num{0.173} &  & \num{-0.175} & \num{0.242} &  &  &  & \num{0.033} \\
& (\num{0.082}) * &  & (\num{0.148}) & (\num{0.093}) ** &  &  &  & (\num{0.063}) \\
Class size & \num{-0.034} & \num{-0.024} & \num{-0.043} & \num{-0.001} & \num{0.085} & \num{0.166} & \num{-0.033} & \num{0.009} \\
& (\num{0.045}) & (\num{0.083}) & (\num{0.130}) & (\num{0.040}) & (\num{0.017}) *** & (\num{0.073}) * & (\num{0.057}) & (\num{0.014}) \\
Class size (squared) & \num{0.001} & \num{0.001} & \num{0.001} & \num{-0.000} & \num{-0.001} & \num{-0.003} & \num{0.001} & \num{-0.000} \\
& (\num{0.001}) & (\num{0.002}) & (\num{0.003}) & (\num{0.001}) & (\num{0.000}) *** & (\num{0.001}) * & (\num{0.001}) & (\num{0.000}) \\
Student–teacher ratio & \num{-0.011} & \num{0.001} & \num{-0.001} & \num{-0.005} & \num{-0.003} & \num{0.035} &  & \num{-0.004} \\
& (\num{0.004}) ** & (\num{0.004}) & (\num{0.005}) & (\num{0.004}) & (\num{0.004}) & (\num{0.030}) &  & (\num{0.001}) *** \\
Teachers with Master's (\%) & \num{0.070} & \num{-0.021} & \num{-0.017} & \num{0.127} & \num{-0.267} & \num{0.048} &  & \num{0.740} \\
& (\num{0.065}) & (\num{0.051}) & (\num{0.142}) & (\num{0.068}) + & (\num{0.181}) & (\num{0.317}) &  & (\num{0.137}) *** \\
Poor School Infrastructure & \num{-0.099} & \num{-0.076} & \num{0.070} & \num{0.095} & \num{-0.069} & \num{0.133} & \num{0.239} & \num{0.042} \\
& (\num{0.101}) & (\num{0.127}) & (\num{0.290}) & (\num{0.119}) & (\num{0.081}) & (\num{0.204}) & (\num{0.216}) & (\num{0.050}) \\
Task performance & \num{0.059} & \num{0.058} & \num{0.070} & \num{0.044} & \num{0.033} & \num{0.034} & \num{0.034} & \num{0.073} \\
& (\num{0.011}) *** & (\num{0.017}) *** & (\num{0.021}) *** & (\num{0.012}) *** & (\num{0.006}) *** & (\num{0.019}) + & (\num{0.013}) ** & (\num{0.009}) *** \\
Average school SES & \num{0.679} & \num{0.666} & \num{0.597} & \num{0.514} & \num{0.621} & \num{0.713} & \num{0.442} & \num{0.311} \\
& (\num{0.105}) *** & (\num{0.123}) *** & (\num{0.275}) * & (\num{0.102}) *** & (\num{0.039}) *** & (\num{0.092}) *** & (\num{0.129}) *** & (\num{0.043}) *** \\
Private school &  &  &  &  & \num{-0.075} &  & \num{0.226} & \num{0.282} \\
&  &  &  &  & (\num{0.062}) &  & (\num{0.224}) & (\num{0.074}) *** \\
Num.Obs. & \num{4543} & \num{2323} & \num{1417} & \num{3751} & \num{11372} & \num{1853} & \num{3268} & \num{5515} \\
R2 Marg. & \num{0.185} & \num{0.150} & \num{0.116} & \num{0.133} & \num{0.340} & \num{0.323} & \num{0.253} & \num{0.381} \\
R2 Cond. & \num{0.330} &  & \num{0.263} & \num{0.284} & \num{0.531} &  & \num{0.402} & \num{0.486} \\
\end{talltblr}

}

\end{table}%

When it comes to net educational productivity, Moscow Region and Moscow
City come out on top. Their schools seem to do particularly well when it
comes to turning classroom time into measurable learning. On the flip
side, Hong Kong, which usually scores very well overall in PISA, shows a
much smaller gain per school year. That effect may arise either as a
consequence of the overall high levels of reading proficiency, which
make marginal gains smaller in magnitude. Notably divergent results
observed in learning gains across the three Chinese territories, namely,
BSJZ, Hong Kong, and Macao, point at potential differences in the
education policy, curriculum, education finance, or resources at the
subnational level between the cities of mainland China and special
administrative regions of Hong Kong and Macao. The same argument can be
applied to Russia, where despite federal governance and autonomous
status of certain administrative units, education system appears to be
rather centralized. While the country overall performs moderately,
learning gains in Tatarstan substantially fall behind from the ones in
Moscow region and Moscow City. Finally, Brazil shows a surprisingly high
learning gain per school year given its lower average PISA scores. This
could imply that while learning levels are low, the year-on-year
improvements are relatively strong, perhaps because lower-performing
systems have more room for growth.

\subsection{Magnitude of School-Level
Factors}\label{magnitude-of-school-level-factors}

To better understand the effectiveness of education systems, it is
essential to examine the factors associated with advanced cognitive
skills at the country levels. The analysis in
Section~\ref{sec-learn-gains} reveals that the average learning gains
per academic year substantially vary across the studied countries and
territories. These learning gains give us a meaningful and more
intuitive way to interpret the regression coefficients of other
variables. If to look at these regression coefficients through the lens
of learning gains, they can be interpreted not in standard deviations,
but in terms of equivalent years of schooling (EYOS), providing a more
policy-oriented approach (Evans \& Yuan, 2019). In that term, a
coefficient of school year would provide the amount of learning under
the ``business-as-usual'' scenario. For instance, if the school year
effect corresponds to 0.2 standard deviations and the regression
coefficient for the share of teachers with a Master's degree accounts
for 0.05 standard deviations, a 1\% increase in the proportion of
qualified teachers would be equivalent to a learning gain of
approximately one-quarter of a year of schooling. While the original
regression coefficients are presented in Table~\ref{tbl-reg-mem},
Figure~\ref{fig-eyos} converts these coefficients into the corresponding
effects with relation to the school year gains.

\begin{figure}

\centering{

\includegraphics{manuscript_files/figure-pdf/fig-eyos-1.pdf}

}

\caption{\label{fig-eyos}Effect of covariates in Equivalent Years of
Schooling (EYOS), relative to the learning gain from one school year,
results of the mixed-effects models}

\end{figure}%

The effect of school location is quite pronounced in Russia overall and
in its autonomous region of Tatarstan. On average across the country,
schools located in a city of a large city have the effect of almost 0.7
EYOS in comparison to rural schools. In the Republic of Tatarstan, the
gap is even higher, with the urban schools producing the advantage in
cognitive skills that is more than 1 year of schooling in comparison to
the rural schools. While the urban landscapes of Moscow City, BSJZ, Hong
Kong and Macao, obviously, do not allow for drawing such a comparison,
the effect of area was not significant in Brazil, which potentially can
be explained by the overlap between the spatial and socio-economic
inequalities in the country.

Findings with respect to teacher qualification appear to be somewhat
surprising. Measured as the share of teachers holding at least a
Master's degree or equivalent, this variable produces statistically
significant effects only in Tatarstan and Brazil. In Tatarstan, an
increase in the number of qualified teachers by 2\% would result in more
than 0.5 EYOS. However, in Brazil this would translate into as much as
2.3 EYOS. This highlights unique opportunities for boosting learning
through investment in teachers' qualifications. The potential
explanation for the insignificant effects in urban settings could
potentially refer to the fact that, for example, in Russia it is quite
common for a teacher to have an advanced university degree. As Tatarstan
sample includes the region's capital, Kazan, and other rural schools,
the effects that are averaged across the country overall and in such
areas as Moscow City and Moscow Region (the capital and its geographic
proximity), might be pronounced when it comes to this specific
administrative unit. While the variable on teachers with Master's was
not included in Macao data, insignificant effects Hong Kong could arise
due to the overall high level of teacher qualifications, as almost half
of teachers in the sampled schools of Hong Kong hold a Master's degree.
A more interesting conclusion refers to BSJZ. Although the share of
teachers holding a Master's degree in BSJZ exhibits quite high
statistical variation with respect to the mean (SD ≈ 0.14 around a mean
of 0.14), this variation does not translate into measurable differences
in student learning outcomes. This likely reflects the fact that higher
qualifications do not necessarily correspond to higher pedagogical
quality or greater instructional variation within this highly
standardized system. In other words, the effect could potentially denote
dominance of systemic factors (standardized curricula and strong
accountability) and the limited link between formal degrees of the
teaching staff and classroom instructional quality.

The effects of student-teacher ratio were found to be significant in the
country-level samples of Russia and Brazil. For example, in Russia a
10-student increase in the student--teacher ratio is associated with a
reduction of about 0.42 equivalent years of schooling. This effect is
less pronounced in Brazil, where increase in 10 students per teacher
would translate into the losses equivalent to 1.4 years of schooling.

Interesting effects were observed with respect to class size, showing
significance in BSJZ and Hong Kong. Importantly, class size establishes
a curvilinear relationship with quality of learning, as both the
coefficient itself and its quadratic terms produce statistically
significant effects. While initially increasing the number of students
per class improves learning, its effect diminishes after reaching a
certain saturation point.

The information on school ownership was collected for the samples of
BSJZ, Hong Kong, Macao, and Brazil. In the case of Hong Kong, the
fixed-effect model matrix was rank-deficient, and the school ownership
variable was automatically dropped during model convergence due to a
lack of within-group variation. For BSJZ and Macao, private schools did
not show statistically significant effects on learning. However, the
effect was very substantial in Brazil, where enrollment in private
school leads to learning gains equivalent to almost 0.9 years of
schooling.

Finally, the self-reported measures of school infrastructure collected
from principals did not exhibit a statistically significant association
with learning outcomes. This finding may indicate that the subjective
nature of the indicator introduces measurement error or social
desirability bias, thereby obscuring its true relationship with
instructional quality and student performance.

\subsection{The Role of Socio-Demographic Factors: Exogenous Effects
Matter}\label{the-role-of-socio-demographic-factors-exogenous-effects-matter}

While socio-demographic factors that describe individual characteristics
are not policy-sensitive and malleable and are associated with exogenous
variation in cognitive skills, understanding their role in shaping
learning is critical in the context of analyzing productivity of
education systems. As such, the analysis identifies gender of a child as
a very strong factor associated with learning. Its effect is
statistically significant across all samples with the exception of
Macao. To be more concrete, being male produces substantially negative
effects on learning, identifying the disadvantage faced by boys in their
cognitive skill gains. The magnitude of the gender effect is the most
pronounced in Tatarstan, where the average loss of boys accounts for
0.86 EYOS. These effects are less pronounced in Moscow Region, Russia,
and Hong Kong samples, yet they account more than 0.5 EYOS. For BSJZ,
Brazil, and Moscow City they range between 0.2 and 0.4 EYOS. These
findings underscore substantial disadvantage of boys in human capital
gains by the end of compulsory education.

Language minority status is another dimension that identifies
vulnerability in the acquisition of cognitive skills. Its effect is
statistically significant across all selected samples with the exception
of Hong Kong. In Macao, BSJZ, Moscow City, Moscow Region, and Russia
overall, being a language minority results in the loss of learning
equivalent to more than 1 year of schooling. In Brazil and Tatarstan
samples it is above 0.5 EYOS. This finding identifies a necessity of
integrating language minorities, who often are indigenous populations of
certain areas within studied territories, and creating targeted
interventions directed at these groups in terms of providing support for
their learning.

Finally, socio-economic status produces advantage in gaining cognitive
skills, with the effect being significant in comparison to the students
from the bottom 40\% by SES in all territories with the exception of
Hong Kong. Russian national sample is the one where advantage of the
students from the families at the top 10\% of socio-economic ladder is
the most pronounced: it results in more than 0.75 EYOS in comparison to
the children from the poorest families. The effects in BSJZ, Moscow
City, Brazil, and Moscow Region are all above 0.5 of the school year.
The lowest magnitude of socio-economic advantage is found in the sample
of Macao students, where the effect is still substantially large to be
ignored and equivalents 0.26 years of schooling.

\subsection{Effect of Non-Cognitive
Skills}\label{effect-of-non-cognitive-skills}

As this research uses motivation to master tasks as a proxy of
non-cognitive skills, the models confirm the existing agenda in the
economics of education literature that personality traits are
productive. As it becomes increasingly evident that non-cognitive skills
are malleable, accounting for these traits in the education production
function offers a new pathway where learning can be improved. The
findings suggest that motivation to master tasks produces significant,
positive, and substantial effects on learning across all studied
samples. Its effect is the most pronounced in Russia and Brazil, where
an increase in task mastery by 1 standard deviation results in gains
equivalent to more than 0.2 years of schooling. These effects are
somewhat lower, in the range of 0.15-0.20 EYOS, in such territories as
Moscow Region, Tatarstan, Hong Kong, and Moscow City. Finally, while the
lowest effects of motivation are observed in Macao and BSJZ, yielding a
gain of 0.10 EYOS. Although this is the smallest effect observed, it is
sufficiently high, marking that the influence of personality remains
consistently significant and high across all countries and territories
examined.

\subsection{SES of School: Peer-Effects Play a
Role}\label{ses-of-school-peer-effects-play-a-role}

Lastly, the social environment in which learning occurs exerts a
powerful influence on student achievement. In the model, peer effects
are approximated by the average ESCS index at the school level,
calculated from the individual ESCS values of enrolled students. This
indicator captures the broader socioeconomic context of the school ---
reflecting not only the material resources available but also the
prevailing norms, aspirations, and learning culture that shape students'
engagement and motivation.

Though the effect of the average school ESCS varies across samples, even
at its lowest bound, it is staggering. A one-standard deviation increase
in the average school-level ESCS is associated with learning gains
equivalent to approximately 0.95 years of schooling in Brazil, over 2
years in Russia and Tatarstan, and more than 3.5 years in Hong Kong.
These results point to the critical role of peer composition in shaping
educational outcomes.

Even after controlling for individual socio-economic status, students
benefit substantially from being surrounded by peers from wealthier
backgrounds. This suggests that the collective advantages of attending a
high-SES school amplify individual learning beyond what personal
background alone would predict. Conversely, students in schools with
high concentrations of socio-economically disadvantaged peers may
experience compound disadvantages, as resource constraints, limited
aspirations, and peer disengagement reinforce each other. Furthermore,
it means that schools with high concentrations of wealthy students are
significantly better at producing cognitive skills, even when students
from poor backgrounds receive the same curriculum, hours of teaching, or
instruction quality.

The persistence of such large peer effects underscores a fundamental
equity challenge: learning outcomes are not determined solely by
individual effort or instructional quality but also by the
socio-economic composition of the classroom. In practice, this means
that inequalities in the distribution of students across schools, which
occur through residential segregation, selective admissions, or other
mechanisms, further translate into inequalities in human capital
acquisition.

\subsection{Disparities in Learning Gains due to Family
SES}\label{disparities-in-learning-gains-due-to-family-ses}

A supplementary research question in this study examines whether
learning gains are heterogeneous with respect to students'
socio-economic backgrounds. The second bloc of mixed-effects models
(outlined in Section~\ref{sec-app6}) incorporates a random slope for
grade by ESCS percentile groups, enabling an assessment of how schooling
productivity varies across students from different socio-economic
strata. The estimated coefficients are presented in Figure~\ref{fig-m2}.

The results indicate that SES does indeed contribute to significant
variations in the amount of cognitive skills gained by students.
However, the extent of these variations is highly contingent on the
socio-economic and cultural context of each country or territory.
Specifically, in Hong Kong and Macao students from the bottom 40\% of
the socio-economic distribution exhibit same learning gains as those
from the middle 50\% and top 10\%, holding it at about 20\% and 30\% of
SD respectively.

Brazil, on the contrary, is the only country where skills gained by a
student in one academic year are dramatically higher if they come from a
wealthier family. As such, while students from the bottom 40\% of
families gain 31\% of SD per academic year, the students from the top
10\% families gain 43\% of SD in learning. In this sense, the effects of
gross educational productivity with respect to general volumes of
learning match those of net educational productivity, i.e., amount of
learning acquired in a unit of time (academic year).

This pattern does not hold true for other territories. In BSJZ, Russia,
and Russian administrative units, while the poorest students have less
cognitive skills, they learn more in academic year. This result can be
explained by the floor effect, when overall higher volumes of learning
observed among students from the wealthier families result in lower
learning gains. In other words, while overall higher cognitive skills
are associated with higher SES, diminishing returns come to play once we
place the amount of skills gained along the socio-economic gradient.

\begin{figure}

\centering{

\includegraphics{manuscript_files/figure-pdf/fig-m2-1.pdf}

}

\caption{\label{fig-m2}Learning Gains Over 1 Academic Year by
Country/Territory and Socio-Economic Percentile Group of a Student
Family, Coefficients of Mixed-Effects Model}

\end{figure}%

\subsection{Disparities in Learning Gains due to School
SES}\label{disparities-in-learning-gains-due-to-school-ses}

Just like socio-economic background of a student has impact on both
their learning volumes overall and their learning gains, so do peer
effects expressed in the socio-economic environment in schools. The fact
that schools with the higher share of wealthier students produce more
cognitive skills even with the same curricula makes it important to
explore the differences in learning gains associated with the
concentration of poorer or wealthier students in certain schools. Do
children who study in schools where the majority of students come from
economically disadvantaged families learn less per academic year than
those who study in wealthier schools? In other words, at this stage the
research explores the question of whether concentration of students with
certain socio-economic backgrounds results in differences in learning
gains per academic year. The results of the third block of mixed-effects
models are shown on Figure~\ref{fig-m3} (full models are referenced in
Section~\ref{sec-app7}).

The findings of the third model are substantially more consistent across
territories, confirming that the more wealthy students a school has, the
higher the learning gains per academic year.

\begin{figure}

\centering{

\includegraphics{manuscript_files/figure-pdf/fig-m3-1.pdf}

}

\caption{\label{fig-m3}Learning Gains Over 1 Academic Year by
Country/Territory and Socio-Economic Percentile Group of School,
Coefficients of Mixed-Effects Model}

\end{figure}%

\section{Discussion}\label{discussion}

This study examines the formation of cognitive skills, approximated by
reading scores, in secondary education across Brazil, Russia, and
selected Chinese administrative units. While differences in educational
outcomes between countries are well documented, this analysis moves
beyond describing disparities in skills as the output of education
process and focuses instead on the inputs and mechanisms that drive
school productivity, as well as on how these inputs produce
heterogeneous effects on learning. The discussion below focuses on three
interrelated mechanisms identified by the study: institutional
stratification reflected in between-school variance, the role of peer
composition in shaping learning opportunities, and the complementary
function of non-cognitive skills in skill formation.

The results highlight that though schools are expected to play central
role in equipping children with cognitive skills, the magnitude of
school-level factors in explaining variation in learning shows
substantial heterogeneity both between and within countries. The
differences observed in the share of between-school variance in learning
range from approximately 15\% in Moscow City to over 50\% in BSJZ,
illustrating how the structure of education systems shapes the relative
role of schools in skill formation. Previous literature has shown that
higher between-school variance is often associated with greater
institutional stratification, including early tracking, selective
admissions, and unequal resource allocation between schools (Hanushek \&
Woessmann, 2011). In the context of the findings produced by this study,
it suggest that the Chinese education system exhibits higher
institutional stratification than the Russian one, with Brazil in
between. Importantly, low between-school variance in Moscow City does
not imply that schools are not important for learning. Given the high
levels of gross educational productivity in Moscow, low between-school
variation in cognitive skills comes primarily not from the institutional
heterogeneity but from factors within schools, often linked to
individual or household characteristics. The case of Moscow City
exemplifies how high quality of educational output can coexist with low
between-school variance, reflecting that cognitive skills are more
evenly distributed across schools.

Further, the findings of this study highlight significant spatial
inequalities in the production of cognitive skills within and across
countries, particularly for China and Russia, where different
administrative units were analyzed separately. Previous analysis on the
Russian PISA data underscored the same result, highlighting that OECD
countries produce smaller variation in learning outcomes than Russian
regions (Adamovich et al., 2019). These spatial disparities reflect the
uneven distribution of resources and institutional capacities across
regions. In stratified systems, territorial differences in school
quality compound socio-economic inequalities, resulting in
geographically uneven distribution of cognitive achievement within
countries.

The strong impact of socio-economic composition of students at the
school level on cognitive achievement underscores the power of peer
effects, which can outweigh the benefits of an additional year of
schooling. Schools with higher proportion of affluent students are more
likely to foster better development of cognitive skills, while schools
with a substantial concentration of economically disadvantaged students
tend to produce lower skill levels despite similar curricula or teacher
qualification. This pattern is aligned with the so-called Matthew effect
in education (Bonoli et al., 2017; Rigney, 2010; Stanovich, 2009;
Walberg \& Tsai, 1983). The term was coined out by American sociologist
R. Merton (Merton, 1968, 1988) and it suggests that cumulative advantage
begets further advantage. With that respect, policies need to address
the inequalities in learning gains by focusing on the needs of the
poorest students.

Furthermore, these findings are in many ways in line with the existing
human capital literature. According to a widely accepted growth model of
human capital accumulation, individual productivity measured by earnings
is a function of social interaction with others, suggesting that an
individual learns more when they interact with more productive people
(Lucas, 2015).

Overall this highlights the role of schools in reproducing, rather than
mitigating, socio-economic inequality. As was rightly mentioned, schools
function as equalizers as long as they mix children from different
socio-economic backgrounds and provide integrated peer
groups(Agostinelli et al., 2020). On the contrast, schools with great
concentration of students belonging to a certain socio-economic strata
instead of being ``great equalizer'' rather reinforce inequality,
contributing to a segregated education system that mirrors broader
social divisions.

Finally, the analysis underscores a pivotal role of motivation to
achieve and master tasks in shaping educational productivity. Although
personality traits were initially considered exogenous to educational
outcomes (Bowles, 1970), advances in psychology and economics have
highlighted the significant returns to these traits. Classified in
economics as non-cognitive skills due to their malleability in early
childhood, personality traits have been shown to exert substantial
influence on a wide range of socio-economic outcomes (Borghans et al.,
2008).

The findings of this study on the potential of task mastery to boost
learning are broadly consistent with previous research on the role of
non-cognitive skills in academic performance. It is also known that
development of task mastery offers a promising path to boost cognitive
skills, particularly among economically disadvantaged students
(Avanesian et al., 2022). Overall, the results of the analysis in this
study highlight that integrating non-cognitive skills into discussions
of educational productivity, both as part of the human capital
production function and as a factor sensitive to policy intervention
(Heckman et al., 2006), is essential for advancing educational theory
and practice. Furthermore, they follow a well-known ``skills beget
skills'' logic, when the stock of skills in non-cognitive domain results
in advances in gaining cognitive skills (Cunha et al., 2006; Cunha et
al., 2010; Cunha \& Heckman, 2007).

Taken together, these findings reveal how institutional heterogeneity,
peer composition, and individual non-cognitive skills interact to shape
cognitive skills, both in gross and net productivity terms. Addressing
educational inequality therefore requires a dual focus. On the one hand,
system-level reforms are needed to reduce spatial and peer segregation
observed on the supply side of education. On the other hand, targeted
interventions that strengthen capacity of students, particularly from
the disadvantaged backgrounds, to benefit from learning refer to
measures aimed at demand side of education systems. This combined
approach is essential for fostering both equity and efficiency in
education systems.

\section{Research Limitations}\label{research-limitations}

The study is subject to several limitations stemming from both data
constraints and the econometric verification strategy employed. First,
the data used in this analysis are cross-sectional, and the absence of
longitudinal records on individual learning outcomes restricts the
estimation of learning gains to a set of specific assumptions. Although
prior economic literature has approximated learning gains over one
academic year by relying on academic grades, it is important to note
that having multiple records per student would enable a more precise
estimation of these gains.

Further, the lack of nationally representative data on China does not
allow for drawing meaningful comparison between Brazil, China, and
Russia. The availability of data only for selected urban centers does
not allow, on the one hand, to have the full picture of cognitive skill
formation in urban settings overall. However, and more importantly, it
is impossible to estimate the role of urban-rural disparities and role
of spatial segregation in learning.

Additionally, while efforts were made to balance the selection of
variables within the model, data limitations, particularly regarding
missingness, precluded the inclusion of certain relevant factors, such
as school infrastructure indicators (e.g., the proportion of computers
with internet access). Furthermore, the absence of data on the
proportion of teachers holding a Master's degree or equivalent in Macao
necessitated the exclusion of this administrative unit from the study.

Another limitation concerns the estimation of peer effects using average
measures. In practice, this approach captures only the concentration of
students in schools based on family wealth, without providing insights
into actual social cohesion or networks that may more accurately reflect
the mechanisms through which peer interactions influence learning
outcomes.

Lastly, a common issue with education production functions lies in the
difficulty of interpreting results in a strictly causal manner. Although
efforts were made to address endogeneity at both the school and country
levels by adopting random effect terms in the model, this approach
cannot be considered sufficient to fully establish causal relationships.
Nevertheless, the findings can serve as a valuable guide for informing
educational policies and programs.

\section{Conclusions}\label{conclusions}

This study estimates cognitive skill gains across Brazil, China, and
Russia using an education production function approach. By adopting a
mixed-effects model, which separates variation at both the country and
school levels, the paper explores the influence of endogenous and
exogenous factors at the individual, household, and school levels on
educational productivity. The results demonstrate that inequality in
human capital production persists not only between countries but also
within them. Specifically, the study identifies the significant impact
of socio-economic factors, both at the individual level and through peer
effects at the school level.

The findings confirm that students from the poorest households
consistently fall behind in learning, and schools with a high
concentration of disadvantaged students tend to produce less skills.
This has far-reaching implications for the reproduction of poverty
patterns and imposes limitations on intergenerational social mobility.
The fact that students exposed to the same curricula, mode of
instruction, and teachers of the same qualification can accumulate less
human capital simply due to attending schools with a high concentration
of disadvantaged peers underscores the role of education systems in
perpetuating socio-economic segregation.

To address this issue, governments must implement targeted policies,
allocate additional resources, and pursue interventions that
specifically support disadvantaged schools and students. Particular
attention should be given to non-cognitive skills, such as fostering
motivation to achieve and master tasks, which plays a crucial role in
narrowing the achievement gap.

Future research should expand the sample of studied countries and
territories to further explore the relationship between learning volumes
and gains, particularly to determine whether diminishing returns are at
play in the production of educational outcomes.

\section{\texorpdfstring{\textbf{Data Availability
Statement}}{Data Availability Statement}}\label{data-availability-statement}

To carry out the analysis, the data of the PISA 2018 were used. The
datasets are publicly available on the website:
\url{https://www.oecd.org/en/about/programmes/pisa/pisa-data.html}.

The R codes for the analysis can be found via the following link:\\
\url{https://github.com/karavan88/EduProdFunctionPISA}.

\section{References}\label{references}

\phantomsection\label{refs}
\begin{CSLReferences}{1}{0}
\bibitem[\citeproctext]{ref-hsepisa2019}
Adamovich, K., Kapuza, A., Zakharov, A., \& Froumin, I. (2019).
\emph{Russia results in math, reading, and science in pisa 2018 and what
they say about education in the country}. Higher School of Economics.
\url{https://ioe.hse.ru/pubs/share/direct/409673299.pdf}

\bibitem[\citeproctext]{ref-agostinelli2020}
Agostinelli, F., Doepke, M., Sorrenti, G., \& Zilibotti, F. (2020).
\emph{When the great equalizer shuts down: Schools, peers, and parents
in pandemic times}. \url{https://doi.org/10.3386/w28264}

\bibitem[\citeproctext]{ref-altinok2017}
Altinok, N., \& Aydemir, A. (2017). Does one size fit all? The impact of
cognitive skills on economic growth. \emph{Journal of Macroeconomics},
\emph{53}, 176--190. \url{https://doi.org/10.1016/j.jmacro.2017.06.007}

\bibitem[\citeproctext]{ref-avanesian2022}
Avanesian, G., Borovskaya, M., Ryzhova, V., Kirik, V., Egorova, V., \&
Bermous, A. (2022). Can we improve learning outcomes of schoolchildren
from the poorest families by investing into their non-cognitive skills?
Causal analysis using propensity score matching. \emph{Voprosy
Obrazovaniya / Educational Studies Moscow}, \emph{1}, 13--53.
\url{https://doi.org/10.17323/1814-9545-2022-1-13-53}

\bibitem[\citeproctext]{ref-howmuch2021}
Avvisati, F., \& Givord, P. (2021a). \emph{How much do 15-year-olds
learn over one year of schooling? An international comparison based on
PISA}. \url{https://doi.org/10.1787/a28ed097-en}

\bibitem[\citeproctext]{ref-thelear2021}
Avvisati, F., \& Givord, P. (2021b). \emph{The learning gain over one
school year among 15-year-olds}.
\url{https://doi.org/10.1787/d99e8c0a-en}

\bibitem[\citeproctext]{ref-avvisati2023}
Avvisati, F., \& Givord, P. (2023). The learning gain over one school
year among 15-year-olds: An international comparison based on PISA.
\emph{Labour Economics}, \emph{84}, 102365.
\url{https://doi.org/10.1016/j.labeco.2023.102365}

\bibitem[\citeproctext]{ref-bates2015}
Bates, D., Machler, M., Bolker, B., \& Walker, S. (2015). Fitting Linear
Mixed-Effects Models Using lme4. \emph{Journal of Statistical Software},
\emph{67}(1). \url{https://doi.org/10.18637/jss.v067.i01}

\bibitem[\citeproctext]{ref-bonoli2017}
Bonoli, G., Cantillon, B., \& Lancker, W. V. (2017). \emph{Social
investment and the matthew effect}. Oxford University Press.
\url{https://doi.org/10.1093/oso/9780198790488.003.0005}

\bibitem[\citeproctext]{ref-borghans2008}
Borghans, L., Duckworth, A. L., Heckman, J. J., \& Weel, B. ter. (2008).
The Economics and Psychology of Personality Traits. \emph{Journal of
Human Resources}, \emph{43}(4), 972--1059.
\url{https://doi.org/10.1353/jhr.2008.0017}

\bibitem[\citeproctext]{ref-bowles1970}
Bowles, S. (1970). Towards an educational production function. In W. L.
Hansen (Ed.), \emph{Education, income, and human capitals} (pp. 11--70).
National Bureau of Economic Research (NBER).

\bibitem[\citeproctext]{ref-bowles1968}
Bowles, S., \& Levin, H. M. (1968). The determinants of scholastic
achievement-an appraisal of some recent evidence. \emph{The Journal of
Human Resources}, \emph{3}(1), 3. \url{https://doi.org/10.2307/144645}

\bibitem[\citeproctext]{ref-brown1975}
Brown, B. W., \& Saks, D. H. (1975). The Production and Distribution of
Cognitive Skills within Schools. \emph{Journal of Political Economy},
\emph{83}(3), 571--593. \url{https://doi.org/10.1086/260341}

\bibitem[\citeproctext]{ref-buchholz2022}
Buchholz, J., Cignetti, M., \& Piacentini, M. (2022). \emph{Developing
measures of engagement in PISA}.
\url{https://doi.org/10.1787/2d9a73ca-en}

\bibitem[\citeproctext]{ref-caponera2016}
Caponera, E., Sestito, P., \& Russo, P. M. (2016). The influence of
reading literacy on mathematics and science achievement. \emph{The
Journal of Educational Research}, \emph{109}(2), 197--204.
\url{https://doi.org/10.1080/00220671.2014.936998}

\bibitem[\citeproctext]{ref-cunha2007}
Cunha, F., \& Heckman, J. (2007). The Technology of Skill Formation.
\emph{American Economic Review}, \emph{97}(2), 31--47.
\url{https://doi.org/10.1257/aer.97.2.31}

\bibitem[\citeproctext]{ref-cunha2006}
Cunha, F., Heckman, J. J., Lochner, L., \& Masterov, D. V. (2006).
\emph{Chapter 12. Interpreting the evidence on life cycle skill
formation} (pp. 697--812). Elsevier.
\url{https://doi.org/10.1016/s1574-0692(06)01012-9}

\bibitem[\citeproctext]{ref-cunha2010}
Cunha, F., Heckman, J., \& Schennach, S. (2010). Estimating the
Technology of Cognitive and Noncognitive Skill Formation.
\emph{Econometrica}, \emph{78}(3), 883--931.
\url{https://doi.org/10.3982/ecta6551}

\bibitem[\citeproctext]{ref-dannemann2019}
Dannemann, B. C. (2019). Peer effects in secondary education: Evidence
from trends in mathematics and science study 2015 based on weak-tie
bonds. \emph{Beiträge Zur Jahrestagung Des Vereins Für Socialpolitik
2019: 30 Jahre Mauerfall - Demokratie Und Marktwirtschaft - Session:
Education Economics II}.

\bibitem[\citeproctext]{ref-epple2004}
Epple, D., Figlio, D., \& Romano, R. (2004). Competition between private
and public schools: testing stratification and pricing predictions.
\emph{Journal of Public Economics}, \emph{88}(7-8), 1215--1245.
\url{https://doi.org/10.1016/s0047-2727(02)00187-1}

\bibitem[\citeproctext]{ref-epple2000b}
Epple, D., Newlon, E., \& Romano, R. (2000). \emph{Ability tracking,
school competition, and the distribution of educational benefits}.
\url{https://doi.org/10.3386/w7854}

\bibitem[\citeproctext]{ref-epple2000a}
Epple, D., \& Romano, R. (2000). \emph{Neighborhood schools, choice, and
the distribution of educational benefits}.
\url{https://doi.org/10.3386/w7850}

\bibitem[\citeproctext]{ref-epple1998}
Epple, D., \& Romano, R. E. (1998). Competition between private and
public schools, vouchers, and peer-group effects. \emph{The American
Economic Review}, \emph{88}(1), 33--62.
\url{http://www.jstor.org/stable/116817}

\bibitem[\citeproctext]{ref-evans2019}
Evans, D. K., \& Yuan, F. (2019). \emph{Equivalent Years of Schooling: A
Metric to Communicate Learning Gains in Concrete Terms}. World Bank,
Washington, DC. \url{https://doi.org/10.1596/1813-9450-8752}

\bibitem[\citeproctext]{ref-glewwe2006}
Glewwe, P., \& Kremer, M. (2006). \emph{Chapter 16. Schools, teachers,
and education outcomes in developing countries} (pp. 945--1017).
Elsevier. \url{https://doi.org/10.1016/s1574-0692(06)02016-2}

\bibitem[\citeproctext]{ref-goldhaber1997}
Goldhaber, D. D., \& Brewer, D. J. (1997). Why don't schools and
teachers seem to matter? Assessing the impact of unobservables on
educational productivity. \emph{The Journal of Human Resources},
\emph{32}(3), 505. \url{https://doi.org/10.2307/146181}

\bibitem[\citeproctext]{ref-brempong1991}
Gyimah-Brempong, K., \& Gyapong, A. O. (1991). Production of Education:
Are Socioeconomic Characteristics Important Factors? \emph{Eastern
Economic Journal}, \emph{17}(4), 507--521.

\bibitem[\citeproctext]{ref-hanchane2012}
Hanchane, S., \& Mostafa, T. (2012). Solving endogeneity problems in
multilevel estimation: an example using education production functions.
\emph{Journal of Applied Statistics}, \emph{39}(5), 1101--1114.
\url{https://doi.org/10.1080/02664763.2011.638705}

\bibitem[\citeproctext]{ref-hanushek1979}
Hanushek, E. A. (1979). Conceptual and empirical issues in the
estimation of educational production functions. \emph{The Journal of
Human Resources}, \emph{14}(3), 351.
\url{https://doi.org/10.2307/145575}

\bibitem[\citeproctext]{ref-hanushek1987}
Hanushek, E. A. (1987). \emph{Educational production functions} (pp.
33--42). Elsevier.
\url{https://doi.org/10.1016/b978-0-08-033379-3.50013-9}

\bibitem[\citeproctext]{ref-hanushek2020}
Hanushek, E. A. (2020). \emph{Education production functions} (pp.
161--170). Elsevier.
\url{https://doi.org/10.1016/b978-0-12-815391-8.00013-6}

\bibitem[\citeproctext]{ref-educatio2007}
Hanushek, E. A., \& Woessmann, L. (2007). \emph{Education quality and
economic growth}. \url{https://doi.org/10.1596/978-0-8213-7058-2}

\bibitem[\citeproctext]{ref-hanushek2008}
Hanushek, E. A., \& Woessmann, L. (2008). The Role of Cognitive Skills
in Economic Development. \emph{Journal of Economic Literature},
\emph{46}(3), 607--668. \url{https://doi.org/10.1257/jel.46.3.607}

\bibitem[\citeproctext]{ref-hanushek2011}
Hanushek, E. A., \& Woessmann, L. (2011). \emph{The economics of
international differences in educational achievement} (pp. 89--200).
Elsevier. \url{https://doi.org/10.1016/b978-0-444-53429-3.00002-8}

\bibitem[\citeproctext]{ref-hanushek2015}
Hanushek, E. A., \& Woessmann, L. (2015). \emph{The knowledge capital of
nations: Education and the economics of growth}. The MIT Press.
\url{https://doi.org/10.7551/mitpress/9780262029179.001.0001}

\bibitem[\citeproctext]{ref-heckman2006a}
Heckman, James~J., Stixrud, J., \& Urzua, S. (2006). The Effects of
Cognitive and Noncognitive Abilities on Labor Market Outcomes and Social
Behavior. \emph{Journal of Labor Economics}, \emph{24}(3), 411--482.
\url{https://doi.org/10.1086/504455}

\bibitem[\citeproctext]{ref-khan2007}
Khan, S. R., \& Kiefer, D. (2007). Educational Production Functions for
Rural Pakistan: A Comparative Institutional Analysis. \emph{Education
Economics}, \emph{15}(3), 327--342.
\url{https://doi.org/10.1080/09645290701273590}

\bibitem[\citeproctext]{ref-lucas2015}
Lucas, R. E. (2015). Human Capital and Growth. \emph{American Economic
Review}, \emph{105}(5), 85--88.
\url{https://doi.org/10.1257/aer.p20151065}

\bibitem[\citeproctext]{ref-mcclung1977}
McClung, R. L. (1977). \emph{Identification of an educational production
function for diverse technologies} (CDT-77/2). Center for Development
Technology, Washington University.

\bibitem[\citeproctext]{ref-merton1968}
Merton, R. K. (1968). The Matthew Effect in Science. \emph{Science},
\emph{159}(3810), 56--63.
\url{https://doi.org/10.1126/science.159.3810.56}

\bibitem[\citeproctext]{ref-merton1988}
Merton, R. K. (1988). The Matthew Effect in Science, II: Cumulative
Advantage and the Symbolism of Intellectual Property. \emph{Isis},
\emph{79}(4), 606--623. \url{https://doi.org/10.1086/354848}

\bibitem[\citeproctext]{ref-mincer1958}
Mincer, J. (1958). Investment in Human Capital and Personal Income
Distribution. \emph{Journal of Political Economy}, \emph{66}(4),
281--302. \url{https://doi.org/10.1086/258055}

\bibitem[\citeproctext]{ref-monk1989}
Monk, D. H. (1989). The Education Production Function: Its Evolving Role
in Policy Analysis. \emph{Educational Evaluation and Policy Analysis},
\emph{11}(1), 31--45. \url{https://doi.org/10.3102/01623737011001031}

\bibitem[\citeproctext]{ref-pisa2012019c}
OECD. (2019a). \emph{PISA 2018 Assessment and Analytical Framework}.
\url{https://doi.org/10.1787/b25efab8-en}

\bibitem[\citeproctext]{ref-pisa2012019a}
OECD. (2019b). PISA 2018 results (volume i). In \emph{PISA}.
\url{https://doi.org/10.1787/5f07c754-en}

\bibitem[\citeproctext]{ref-pisa2012019b}
OECD. (2019c). \emph{PISA 2018 Results (Volume II)}.
\url{https://doi.org/10.1787/b5fd1b8f-en}

\bibitem[\citeproctext]{ref-R}
R Core Team. (2021). \emph{R: A language and environment for statistical
computing}. R Foundation for Statistical Computing.
\url{https://www.R-project.org/}

\bibitem[\citeproctext]{ref-rigney2010}
Rigney, D. (2010). Matthew effects in education and culture. In
\emph{The matthew effect: How advantage begets further advantage} (pp.
75--86). Columbia University Press.
\url{http://www.jstor.org/stable/10.7312/rign14948.8}

\bibitem[\citeproctext]{ref-sacerdote2011}
Sacerdote, B. (2011). \emph{Peer Effects in Education: How Might They
Work, How Big Are They and How Much Do We Know Thus Far?} (pp.
249--277). Elsevier.
\url{https://doi.org/10.1016/b978-0-444-53429-3.00004-1}

\bibitem[\citeproctext]{ref-simmons1978}
Simmons, J., \& Alexander, L. (1978). The Determinants of School
Achievement in Developing Countries: A Review of the Research.
\emph{Economic Development and Cultural Change}, \emph{26}(2), 341--357.
\url{https://doi.org/10.1086/451019}

\bibitem[\citeproctext]{ref-stanovich2009}
Stanovich, K. E. (2009). Matthew Effects in Reading: Some Consequences
of Individual Differences in the Acquisition of Literacy. \emph{Journal
of Education}, \emph{189}(1-2), 23--55.
\url{https://doi.org/10.1177/0022057409189001-204}

\bibitem[\citeproctext]{ref-todd2003}
Todd, P. E., \& Wolpin, K. I. (2003). On the Specification and
Estimation of the Production Function for Cognitive Achievement.
\emph{The Economic Journal}, \emph{113}(485), F3--F33.
\url{https://doi.org/10.1111/1468-0297.00097}

\bibitem[\citeproctext]{ref-vittadini2022}
Vittadini, G., Sturaro, C., \& Folloni, G. (2022). Non-Cognitive Skills
and Cognitive Skills to measure school efficiency. \emph{Socio-Economic
Planning Sciences}, \emph{81}, 101058.
\url{https://doi.org/10.1016/j.seps.2021.101058}

\bibitem[\citeproctext]{ref-walberg1983}
Walberg, H. J., \& Tsai, S.-L. (1983). Matthew Effects in Education.
\emph{American Educational Research Journal}, \emph{20}(3), 359--373.
\url{https://doi.org/10.3102/00028312020003359}

\bibitem[\citeproctext]{ref-zhu2021}
Zhu, Y. (2021). Reading matters more than mathematics in science
learning: an analysis of the relationship between student achievement in
reading, mathematics, and science. \emph{International Journal of
Science Education}, \emph{44}(1), 1--17.
\url{https://doi.org/10.1080/09500693.2021.2007552}

\end{CSLReferences}

\newpage

\section{Appendix}\label{appendix}

\subsection{Correlation between learning outcomes}\label{sec-app1}

\includegraphics{manuscript_files/figure-pdf/unnamed-chunk-1-1.pdf}

\subsection{Table of descriptive statistics of the variables of
study}\label{sec-app2}

\begin{table}
\fontsize{12.0pt}{14.0pt}\selectfont
\begin{tabular*}{\linewidth}{@{\extracolsep{\fill}}llrrrrr}
\toprule
Country/Territory & Variable & mean & sd & median & min & max \\ 
\midrule\addlinespace[2.5pt]
B-S-J-Z (China) & Reading Score & 0.60 & 0.83 & 0.67 & -2.48 & 2.87 \\ 
B-S-J-Z (China) & Task Mastery & 0.33 & 0.95 & -0.10 & -2.89 & 1.93 \\ 
B-S-J-Z (China) & Class Size & 38.84 & 8.19 & 38.00 & 13.00 & 53.00 \\ 
B-S-J-Z (China) & Student-Teacher Ratio & 10.65 & 6.17 & 10.02 & 1.00 & 100.00 \\ 
B-S-J-Z (China) & Poor School Infrastructire & 0.08 & 0.27 & 0.00 & 0.00 & 1.00 \\ 
B-S-J-Z (China) & Qualified Teachers (\%) & 0.14 & 0.13 & 0.10 & 0.00 & 0.77 \\ 
B-S-J-Z (China) & School ESCS (avg) & -0.07 & 0.67 & -0.08 & -1.61 & 1.52 \\ 
Brazil & Reading Score & -0.80 & 0.92 & -0.85 & -3.17 & 2.36 \\ 
Brazil & Task Mastery & 0.29 & 1.07 & -0.10 & -2.89 & 1.93 \\ 
Brazil & Class Size & 35.69 & 7.54 & 38.00 & 13.00 & 53.00 \\ 
Brazil & Student-Teacher Ratio & 29.05 & 16.68 & 25.60 & 1.44 & 100.00 \\ 
Brazil & Poor School Infrastructire & 0.18 & 0.38 & 0.00 & 0.00 & 1.00 \\ 
Brazil & Qualified Teachers (\%) & 0.09 & 0.13 & 0.05 & 0.00 & 1.00 \\ 
Brazil & School ESCS (avg) & -0.69 & 0.71 & -0.75 & -3.20 & 1.26 \\ 
Hong Kong & Reading Score & 0.28 & 0.90 & 0.38 & -2.93 & 2.75 \\ 
Hong Kong & Task Mastery & -0.02 & 0.89 & -0.10 & -2.89 & 1.93 \\ 
Hong Kong & Class Size & 27.73 & 5.13 & 28.00 & 13.00 & 38.00 \\ 
Hong Kong & Student-Teacher Ratio & 13.23 & 9.27 & 12.34 & 6.75 & 100.00 \\ 
Hong Kong & Poor School Infrastructire & 0.04 & 0.21 & 0.00 & 0.00 & 1.00 \\ 
Hong Kong & Qualified Teachers (\%) & 0.50 & 0.16 & 0.50 & 0.00 & 0.83 \\ 
Hong Kong & School ESCS (avg) & -0.16 & 0.53 & -0.23 & -1.04 & 1.23 \\ 
Macao & Reading Score & 0.26 & 0.85 & 0.33 & -2.96 & 2.43 \\ 
Macao & Task Mastery & 0.01 & 0.90 & -0.10 & -2.89 & 1.93 \\ 
Macao & Class Size & 29.20 & 6.07 & 28.00 & 13.00 & 43.00 \\ 
Macao & Student-Teacher Ratio & NaN & NA & NA & Inf & -Inf \\ 
Macao & Poor School Infrastructire & 0.09 & 0.29 & 0.00 & 0.00 & 1.00 \\ 
Macao & Qualified Teachers (\%) & NaN & NA & NA & Inf & -Inf \\ 
Macao & School ESCS (avg) & -0.17 & 0.43 & -0.27 & -0.94 & 0.89 \\ 
Moscow City (RUS) & Reading Score & 0.34 & 0.81 & 0.40 & -2.90 & 2.50 \\ 
Moscow City (RUS) & Task Mastery & -0.35 & 0.91 & -0.33 & -2.89 & 1.93 \\ 
Moscow City (RUS) & Class Size & 25.21 & 2.81 & 23.00 & 18.00 & 33.00 \\ 
Moscow City (RUS) & Student-Teacher Ratio & 19.21 & 5.50 & 18.63 & 1.00 & 42.78 \\ 
Moscow City (RUS) & Poor School Infrastructire & 0.04 & 0.20 & 0.00 & 0.00 & 1.00 \\ 
Moscow City (RUS) & Qualified Teachers (\%) & 0.47 & 0.45 & 0.18 & 0.00 & 1.00 \\ 
Moscow City (RUS) & School ESCS (avg) & 0.78 & 0.19 & 0.77 & 0.40 & 1.32 \\ 
Moscow Region (RUS) & Reading Score & -0.11 & 0.85 & -0.08 & -2.55 & 2.50 \\ 
Moscow Region (RUS) & Task Mastery & -0.35 & 0.98 & -0.10 & -2.89 & 1.93 \\ 
Moscow Region (RUS) & Class Size & 25.39 & 3.32 & 23.00 & 13.00 & 33.00 \\ 
Moscow Region (RUS) & Student-Teacher Ratio & 19.74 & 10.92 & 18.38 & 2.70 & 88.85 \\ 
Moscow Region (RUS) & Poor School Infrastructire & 0.07 & 0.25 & 0.00 & 0.00 & 1.00 \\ 
Moscow Region (RUS) & Qualified Teachers (\%) & 0.56 & 0.43 & 0.82 & 0.00 & 1.00 \\ 
Moscow Region (RUS) & School ESCS (avg) & 0.56 & 0.21 & 0.59 & -0.61 & 0.95 \\ 
Russian Federation & Reading Score & -0.17 & 0.87 & -0.15 & -3.04 & 2.39 \\ 
Russian Federation & Task Mastery & -0.34 & 0.93 & -0.10 & -2.89 & 1.93 \\ 
Russian Federation & Class Size & 24.55 & 4.42 & 23.00 & 13.00 & 33.00 \\ 
Russian Federation & Student-Teacher Ratio & 16.70 & 6.29 & 16.77 & 1.85 & 81.52 \\ 
Russian Federation & Poor School Infrastructire & 0.07 & 0.25 & 0.00 & 0.00 & 1.00 \\ 
Russian Federation & Qualified Teachers (\%) & 0.46 & 0.44 & 0.38 & 0.00 & 1.00 \\ 
Russian Federation & School ESCS (avg) & 0.40 & 0.31 & 0.41 & -1.33 & 1.04 \\ 
Tatarstan (RUS) & Reading Score & -0.32 & 0.86 & -0.32 & -3.12 & 2.57 \\ 
Tatarstan (RUS) & Task Mastery & -0.28 & 0.96 & -0.10 & -2.89 & 1.93 \\ 
Tatarstan (RUS) & Class Size & 23.90 & 5.30 & 23.00 & 13.00 & 53.00 \\ 
Tatarstan (RUS) & Student-Teacher Ratio & 14.34 & 8.59 & 14.30 & 1.00 & 100.00 \\ 
Tatarstan (RUS) & Poor School Infrastructire & 0.09 & 0.29 & 0.00 & 0.00 & 1.00 \\ 
Tatarstan (RUS) & Qualified Teachers (\%) & 0.65 & 0.42 & 0.89 & 0.00 & 1.00 \\ 
Tatarstan (RUS) & School ESCS (avg) & 0.38 & 0.29 & 0.38 & -0.95 & 1.14 \\ 
\bottomrule
\end{tabular*}
\end{table}

\subsection{Probability density plots of reading by
territory}\label{sec-app3}

\includegraphics{manuscript_files/figure-pdf/unnamed-chunk-3-1.pdf}

\subsection{Correlation matrix between the variables selected for the
mixed-effects model}\label{sec-app4}

\includegraphics{manuscript_files/figure-pdf/unnamed-chunk-4-1.pdf}

\subsection{Mixed-effects regression models of learning gains by
socio-economic status group, the models with random slope
terms}\label{sec-app6}

\begin{table}
\centering
\begin{talltblr}[         %% tabularray outer open
entry=none,label=none,
note{}={  },
]                     %% tabularray outer close
{                     %% tabularray inner open
colspec={Q[]Q[]Q[]Q[]Q[]Q[]Q[]Q[]Q[]},
hline{2}={1-9}{solid, black, 0.05em},
hline{30}={1-9}{solid, black, 0.05em},
hline{1}={1-9}{solid, black, 0.1em},
hline{33}={1-9}{solid, black, 0.1em},
column{2-9}={}{halign=c},
column{1}={}{halign=l},
}                     %% tabularray inner close
& Russia & Moscow City & Moscow Region & Tatarstan & BSJZ & Hong Kong & Macao & Brazil \\
(Intercept) & \num{-2.180} & \num{-3.082} & \num{-2.938} & \num{-2.551} & \num{-4.251} & \num{-4.007} & \num{-2.664} & \num{-4.150} \\
& (\num{0.553}) *** & (\num{1.215}) * & (\num{1.625}) + & (\num{0.621}) *** & (\num{0.360}) *** & (\num{0.933}) *** & (\num{0.736}) *** & (\num{0.622}) *** \\
School year & \num{0.258} & \num{0.347} & \num{0.359} & \num{0.240} & \num{0.323} & \num{0.199} & \num{0.330} & \num{0.356} \\
& (\num{0.032}) *** & (\num{0.058}) *** & (\num{0.069}) *** & (\num{0.049}) *** & (\num{0.019}) *** & (\num{0.028}) *** & (\num{0.017}) *** & (\num{0.059}) *** \\
Sex: Male & \num{-0.167} & \num{-0.131} & \num{-0.247} & \num{-0.208} & \num{-0.066} & \num{-0.120} & \num{-0.034} & \num{-0.089} \\
& (\num{0.020}) *** & (\num{0.030}) *** & (\num{0.040}) *** & (\num{0.023}) *** & (\num{0.011}) *** & (\num{0.038}) ** & (\num{0.026}) & (\num{0.018}) *** \\
Language minority & \num{-0.284} & \num{-0.467} & \num{-0.418} & \num{-0.128} & \num{-0.476} & \num{-0.024} & \num{-0.610} & \num{-0.216} \\
& (\num{0.047}) *** & (\num{0.081}) *** & (\num{0.096}) *** & (\num{0.034}) *** & (\num{0.079}) *** & (\num{0.054}) & (\num{0.046}) *** & (\num{0.083}) ** \\
Area: Town & \num{0.186} &  & \num{-0.224} & \num{0.156} &  &  &  & \num{0.007} \\
& (\num{0.088}) * &  & (\num{0.192}) & (\num{0.107}) &  &  &  & (\num{0.059}) \\
Area: City/Large City & \num{0.172} &  & \num{-0.177} & \num{0.241} &  &  &  & \num{0.033} \\
& (\num{0.083}) * &  & (\num{0.148}) & (\num{0.093}) ** &  &  &  & (\num{0.063}) \\
Class size & \num{-0.034} & \num{-0.022} & \num{-0.046} & \num{-0.001} & \num{0.085} & \num{0.165} & \num{-0.033} & \num{0.011} \\
& (\num{0.045}) & (\num{0.083}) & (\num{0.131}) & (\num{0.040}) & (\num{0.017}) *** & (\num{0.073}) * & (\num{0.057}) & (\num{0.014}) \\
Class size (squared) & \num{0.001} & \num{0.001} & \num{0.001} & \num{-0.000} & \num{-0.001} & \num{-0.003} & \num{0.001} & \num{-0.000} \\
& (\num{0.001}) & (\num{0.002}) & (\num{0.003}) & (\num{0.001}) & (\num{0.000}) *** & (\num{0.001}) * & (\num{0.001}) & (\num{0.000}) \\
Student–teacher ratio & \num{-0.011} & \num{0.001} & \num{-0.001} & \num{-0.005} & \num{-0.003} & \num{0.035} &  & \num{-0.004} \\
& (\num{0.004}) ** & (\num{0.004}) & (\num{0.005}) & (\num{0.004}) & (\num{0.004}) & (\num{0.030}) &  & (\num{0.001}) *** \\
Teachers with Master's (\%) & \num{0.071} & \num{-0.021} & \num{-0.019} & \num{0.126} & \num{-0.268} & \num{0.051} &  & \num{0.745} \\
& (\num{0.065}) & (\num{0.050}) & (\num{0.142}) & (\num{0.068}) + & (\num{0.181}) & (\num{0.316}) &  & (\num{0.136}) *** \\
Poor School Infrastructure & \num{-0.101} & \num{-0.080} & \num{0.070} & \num{0.095} & \num{-0.069} & \num{0.132} & \num{0.239} & \num{0.041} \\
& (\num{0.101}) & (\num{0.127}) & (\num{0.292}) & (\num{0.119}) & (\num{0.081}) & (\num{0.204}) & (\num{0.215}) & (\num{0.050}) \\
Task performance & \num{0.059} & \num{0.058} & \num{0.072} & \num{0.045} & \num{0.033} & \num{0.034} & \num{0.035} & \num{0.073} \\
& (\num{0.011}) *** & (\num{0.017}) *** & (\num{0.021}) *** & (\num{0.012}) *** & (\num{0.006}) *** & (\num{0.019}) + & (\num{0.013}) ** & (\num{0.008}) *** \\
Average school SES & \num{0.683} & \num{0.674} & \num{0.620} & \num{0.521} & \num{0.621} & \num{0.703} & \num{0.457} & \num{0.316} \\
& (\num{0.105}) *** & (\num{0.123}) *** & (\num{0.275}) * & (\num{0.102}) *** & (\num{0.039}) *** & (\num{0.090}) *** & (\num{0.128}) *** & (\num{0.043}) *** \\
Private school &  &  &  &  & \num{-0.075} &  & \num{0.226} & \num{0.275} \\
&  &  &  &  & (\num{0.062}) &  & (\num{0.224}) & (\num{0.074}) *** \\
Num.Obs. & \num{4543} & \num{2323} & \num{1417} & \num{3751} & \num{11372} & \num{1853} & \num{3268} & \num{5515} \\
R2 Marg. & \num{0.181} & \num{0.112} & \num{0.116} & \num{0.131} & \num{0.387} & \num{0.323} & \num{0.294} & \num{0.366} \\
R2 Cond. &  &  &  &  &  &  &  & \num{0.489} \\
\end{talltblr}
\end{table}

\subsection{Mixed-effects regression models of learning gains by
socio-economic status of students per school, the models with random
slope terms}\label{sec-app7}

\begin{table}
\centering
\begin{talltblr}[         %% tabularray outer open
entry=none,label=none,
note{}={  },
]                     %% tabularray outer close
{                     %% tabularray inner open
colspec={Q[]Q[]Q[]Q[]Q[]Q[]Q[]Q[]Q[]},
hline{2}={1-9}{solid, black, 0.05em},
hline{32}={1-9}{solid, black, 0.05em},
hline{1}={1-9}{solid, black, 0.1em},
hline{34}={1-9}{solid, black, 0.1em},
column{2-9}={}{halign=c},
column{1}={}{halign=l},
}                     %% tabularray inner close
& Russia & Moscow City & Moscow Region & Tatarstan & BSJZ & Hong Kong & Macao & Brazil \\
(Intercept) & \num{-2.513} & \num{-2.340} & \num{-3.823} & \num{-3.095} & \num{-4.329} & \num{-3.382} & \num{-2.616} & \num{-4.360} \\
& (\num{0.531}) *** & (\num{1.265}) + & (\num{1.585}) * & (\num{0.543}) *** & (\num{0.414}) *** & (\num{1.028}) ** & (\num{0.774}) *** & (\num{0.265}) *** \\
School year & \num{0.271} & \num{0.345} & \num{0.384} & \num{0.274} & \num{0.336} & \num{0.223} & \num{0.341} & \num{0.343} \\
& (\num{0.029}) *** & (\num{0.039}) *** & (\num{0.064}) *** & (\num{0.058}) *** & (\num{0.022}) *** & (\num{0.035}) *** & (\num{0.024}) *** & (\num{0.021}) *** \\
Sex: Male & \num{-0.168} & \num{-0.131} & \num{-0.246} & \num{-0.207} & \num{-0.066} & \num{-0.121} & \num{-0.034} & \num{-0.090} \\
& (\num{0.020}) *** & (\num{0.030}) *** & (\num{0.040}) *** & (\num{0.023}) *** & (\num{0.011}) *** & (\num{0.038}) ** & (\num{0.027}) & (\num{0.018}) *** \\
SES: Middle 50\% & \num{0.158} & \num{0.261} & \num{0.143} & \num{0.172} & \num{0.096} & \num{-0.025} & \num{0.048} & \num{0.117} \\
& (\num{0.023}) *** & (\num{0.033}) *** & (\num{0.044}) ** & (\num{0.025}) *** & (\num{0.013}) *** & (\num{0.040}) & (\num{0.026}) + & (\num{0.021}) *** \\
SES: Top 10\% & \num{0.206} & \num{0.213} & \num{0.199} & \num{0.106} & \num{0.206} & \num{-0.008} & \num{0.089} & \num{0.180} \\
& (\num{0.040}) *** & (\num{0.056}) *** & (\num{0.074}) ** & (\num{0.045}) * & (\num{0.023}) *** & (\num{0.065}) & (\num{0.045}) + & (\num{0.037}) *** \\
Language minority & \num{-0.286} & \num{-0.466} & \num{-0.424} & \num{-0.128} & \num{-0.478} & \num{0.002} & \num{-0.606} & \num{-0.219} \\
& (\num{0.047}) *** & (\num{0.081}) *** & (\num{0.097}) *** & (\num{0.034}) *** & (\num{0.079}) *** & (\num{0.056}) & (\num{0.045}) *** & (\num{0.083}) ** \\
Area: Town & \num{0.135} &  & \num{-0.286} & \num{0.129} &  &  &  & \num{0.072} \\
& (\num{0.091}) &  & (\num{0.192}) & (\num{0.108}) &  &  &  & (\num{0.060}) \\
Area: City/Large City & \num{0.139} &  & \num{-0.159} & \num{0.233} &  &  &  & \num{0.131} \\
& (\num{0.084}) + &  & (\num{0.151}) & (\num{0.093}) * &  &  &  & (\num{0.062}) * \\
Class size & \num{0.005} & \num{-0.051} & \num{0.022} & \num{0.038} & \num{0.084} & \num{0.090} & \num{-0.051} & \num{0.015} \\
& (\num{0.046}) & (\num{0.094}) & (\num{0.129}) & (\num{0.040}) & (\num{0.019}) *** & (\num{0.077}) & (\num{0.057}) & (\num{0.014}) \\
Class size (squared) & \num{-0.000} & \num{0.001} & \num{-0.000} & \num{-0.001} & \num{-0.001} & \num{-0.002} & \num{0.001} & \num{-0.000} \\
& (\num{0.001}) & (\num{0.002}) & (\num{0.003}) & (\num{0.001}) & (\num{0.000}) *** & (\num{0.001}) & (\num{0.001}) & (\num{0.000}) \\
Student–teacher ratio & \num{-0.009} & \num{0.001} & \num{-0.001} & \num{-0.004} & \num{-0.005} & \num{0.035} &  & \num{-0.005} \\
& (\num{0.004}) * & (\num{0.005}) & (\num{0.005}) & (\num{0.004}) & (\num{0.004}) & (\num{0.035}) &  & (\num{0.001}) *** \\
Teachers with Master's (\%) & \num{0.060} & \num{-0.032} & \num{0.001} & \num{0.131} & \num{0.137} & \num{0.103} &  & \num{0.769} \\
& (\num{0.065}) & (\num{0.057}) & (\num{0.150}) & (\num{0.068}) + & (\num{0.203}) & (\num{0.344}) &  & (\num{0.139}) *** \\
Poor School Infrastructure & \num{-0.099} & \num{-0.054} & \num{0.075} & \num{0.073} & \num{-0.076} & \num{0.091} & \num{0.168} & \num{-0.012} \\
& (\num{0.102}) & (\num{0.144}) & (\num{0.301}) & (\num{0.119}) & (\num{0.091}) & (\num{0.223}) & (\num{0.217}) & (\num{0.051}) \\
Task performance & \num{0.060} & \num{0.059} & \num{0.070} & \num{0.043} & \num{0.033} & \num{0.034} & \num{0.034} & \num{0.073} \\
& (\num{0.011}) *** & (\num{0.017}) *** & (\num{0.021}) *** & (\num{0.012}) *** & (\num{0.006}) *** & (\num{0.019}) + & (\num{0.013}) ** & (\num{0.009}) *** \\
Private school &  &  &  &  & \num{-0.031} &  & \num{0.226} & \num{0.337} \\
&  &  &  &  & (\num{0.071}) &  & (\num{0.234}) & (\num{0.078}) *** \\
Num.Obs. & \num{4543} & \num{2323} & \num{1417} & \num{3751} & \num{11372} & \num{1853} & \num{3268} & \num{5515} \\
R2 Marg. & \num{0.097} & \num{0.113} & \num{0.119} & \num{0.100} & \num{0.177} & \num{0.087} & \num{0.283} & \num{0.323} \\
\end{talltblr}
\end{table}




\end{document}
